\documentclass[lang=english]{secseminar}
%\documentclass[lang=ngerman]{secseminar}


\title{Elementary Representation Theory}
\author{Fabian Weik}
\course{cs} % cs, se, mcl, msv, bis, simtech, or {my own special one}
\coursename{Hauptseminar Informationssicherheit und Kryptographie}
\coursesemester{WiSe 2022}
% Wenn der akademische Grad mit angegeben wird: Bitte auf korrekte Position achten (angelsächsische Grade nach dem Namen, deutsche Grade vor dem Namen).
\supervisor{Bertram Betreuer, M.Sc.}

\begin{document}

\begin{abstract}
Im Seminar sammeln Sie erste Erfahrungen mit dem Erstellen von wissenschaftlichen Arbeiten.
Wir geben daher einen kurzen Überblick über die Ziele und Ihre Aufgaben beim Anfertigen einer
Seminararbeit.
Anschließend zeigen wir ein Beispiel für eine sinnvolle Struktur der Ausarbeitung.
Wir listen auf, welche Formalia Sie beachten sollten und geben zum Schluss einige wichtige Hinweise,
die Sie beachten sollten.
\end{abstract}

\section{Einleitung}
Ein Ziel des Studiums ist es, Studierenden das wissenschaftliche Arbeiten sowie
das wissenschaftliche Schreiben näher zu bringen. Erste Erfahrungen mit dem
Erstellen von wissenschaftlichen Arbeiten sammeln Studierende in der Regel
beim Verfassen von Seminarausarbeitungen und Abschlussarbeiten. Daher hier ein
kurzer Überblick, auf was beim Erstellen einer Seminarausarbeitung zu achten
ist.

\medskip Diese Kurzeinführung zum Thema Seminarausarbeitung basiert auf \cite{LIB,BOH07,KAR}.


\section{Ziele einer Seminararbeit}
Im Rahmen des Seminars ``Informationssicherheit und Kryptographie'' haben Sie die Aufgabe,
``Expert*in'' zu einem wissenschaftlichem Thema\footnote{Beachten Sie, dass Ihr Thema von Ihrem
Betreuer genauer eingegrenzt und Schwerpunkte definiert werden können.} zu werden. Dies bedeutet
insbesondere, dass
\begin{itemize}
\item Sie Ihr Thema inhaltlich verstehen,
\item Sie sich zu Ihrem Thema eigene Gedanken machen, und
\item Sie sich auch über die vorgegeben Unterlagen hinaus über Ihr Thema informieren.
\end{itemize}


Das Ziel der Seminararbeit ist, den Lesenden dieses fachliche Thema näher zu bringen und zu
erläutern. Folgende Punkte sind dabei insbesondere zu beachten:

\begin{itemize}
\item Ihre Aufgabe besteht nicht ausschließlich in der reinen Wiedergabe eines Themas. Auch die
  Ausarbeitung sollte zeigen, dass Sie sich eigene Gedanken zu Ihrem Thema gemacht haben.
\item Die Zielgruppe Ihrer Ausarbeitung sind Kommiliton*innen mit ähnlichem Wissensstand wie Sie. Das
  heißt, Sie können erwarten, dass Lesende Einführungsveranstaltungen im Bereich Informatik
  (einschließlich der Veranstaltung \emph{Grundlagen der Informationssicherheit}) erfolgreich
  besucht haben.
\item Ihre Ausarbeitung sollte in sich geschlossen sein.  Lesende müssen von Ihnen in die Lage
  versetzt werden, Ihre Ausarbeitung ohne Zuhilfenahme weiterer Literatur zu verstehen. Daher:
  Führen Sie grundlegende Begriffe und Konzepte ein, falls Sie nicht erwarten können, dass Lesende
  mit ihnen vertraut sind. Sollten Sie sich bei einem Aspekt nicht sicher sein, ob Sie deren Kenntnis
  bei den Lesenden voraussetzen können, fragen Sie Ihre Betreuungsperson.
\item Üblicherweise können Sie viel mehr zu einem Thema schreiben, als Sie Platz zur Verfügung
  haben. Es ist Teil Ihrer Aufgabe, das Thema in dem vorgegebenen Rahmen möglichst gut
  darzustellen. Dazu müssen Sie ggf. an einigen Stellen abstrahieren.
\item Vermitteln Sie neben den technischen Fakten und Sachverhalten den Lesenden auch eine Intuition
  für Ihr Thema. Veranschaulichungen, Beispiele und Bilder können --~in Maßen verwendet~-- Lesende
  beim Verstehen Ihres Themas unterstützen.  Finden Sie ein gutes Mittelmaß zwischen oberflächlicher
  Betrachtung und technischen Details.
\item Englische Fachbegriffe sollten nur sehr vorsichtig (oder besser gar nicht) eingedeutscht
  werden. Im Zweifel lieber zu wenig als zu viel eindeutschen bzw. gleich die ganze Ausarbeitung in
  englischer Sprache verfassen.
\end{itemize}

\section{Struktur der Ausarbeitung}
Eine wissenschaftliche Ausarbeitung sollte sinnvoll strukturiert sein. Das heißt, es sollte einen
roten Faden geben, der Lesende sinnvoll durch das Thema führt. Stellen Sie sicher, dass den Lesenden
zu jedem Zeitpunkt der Sinn und Zweck des aktuellen Kapitels klar ist, etwa indem Sie vorher einen
kurzen Überblick geben.

Im Folgenden zeigen wir Ihnen beispielhaft eine übliche Struktur für wissenschaftliche
Ausarbeitungen. Dabei sind Struktur, Reihenfolge, als auch Benennung der Abschnitte \emph{nicht
  strikt festgelegt}. Überlegen und entscheiden Sie selbst, wie Sie Ihr Thema verständlich und
technisch korrekt darstellen können.

\subsection{Kurzfassung / Abstract}
In der Kurzfassung (englisch: Abstract) geben Sie eine möglichst kurze und prägnante Inhaltsangabe
Ihrer Arbeit.
Sie sollte alle wichtigen Schlüsselwörter enthalten und eine ähnliche Struktur wie die Arbeit selber
haben.
Als Faustregel sollte jeder Abschnitt (``section'') in ca. einem Satz zusammengefasst werden.
Eine typische Länge für die Kurzfassung sind 150 bis 200 Wörter.

\subsection{Einleitung}
Eine Einleitung führt Lesende in das Thema der Ausarbeitung bzw. einer wissenschaftlichen Arbeit
ein. Erklären Sie, welches Problem Sie in Ihrer Ausarbeitung behandeln und warum dieses Problem ein
relevantes Problem ist. Vielleicht setzten Sie ihr Thema in einen sinnvollen Kontext. Geben Sie
einen Über- und Ausblick über die Ausarbeitung. Welcher Abschnitt behandelt welches Thema?

\subsection{Grundlagen}
Führen Sie alle Begriffe, die Ihre Lesergruppe potentiell nicht kennt, ein. Wählen Sie dabei die
passende Detailtiefe: Müssen Sie einen Begriff formal definieren oder reicht es, den Lesenden eine
Intuition zu vermitteln? Beschreiben Sie die Grundlagen prägnant.

\subsection{Hauptteil}
Der Hauptteil der Ausarbeitung ist der eigentliche Kern Ihrer Arbeit, in dem Sie ihr Thema den Lesenden
erklären. Der Begriff ``Hauptteil'' bezeichnet hierbei ein Konzept und ist keinesfalls die
Überschrift für diesen Abschnitt. Normalerweise umfasst der Hauptteil ohnehin mehrere Kapitel.

\subsection{Verwandte Arbeiten und Wissenschaftlicher Hintergrund}

Eine Auflistung verwandter Arbeiten ist essentiell für jeden wissenschaftlichen Artikel.\\

\noindent
\emph{Zur Position.}
Bereits in der Einleitung sollte auf die Unzulänglichkeiten verwandter Arbeiten eingegangen werden,
um die eigenen Arbeit zu motivieren.
In kürzeren Ausarbeitung mit einer kleinen Anzahl ähnlicher veröffentlichter wissenschaftlicher
Arbeiten kann man die verwandten Arbeiten alternativ schon komplett in der Einleitung abhandeln.
In längeren Ausarbeitungen oder sobald der Umfang der verwandten Literatur einen eigenen Abschnitt
rechtfertigt, sollte jedoch ein eigener Abschnitt "Related Work/Verwandte Arbeiten" angelegt werden.
In jedem Fall sollte das Thema "Verwandte Arbeiten" für Lesende erkennbar und nachvollziehar
dargestellt werden.
Der eigene Abschnitt kann entweder vor den eigentlichen Beitrag gestellt werden (also der Abschnitt
nach der Einleitung), um die Verbesserungen der Arbeit gegenüber anderen Arbeiten zu betonen.
Alternativ kann dieser Abschnitt auch erst am Ende eines Papers vor Zusammenfassung und Ausblick
auftauchen.\\

\noindent
\emph{Zum Inhalt}.
In den Abschnitt "Related Work" gehört beispielsweise ein bekannter Überblicks-Artikel zum
Themengebiet bzw. ein standard-referenziertes Buch.
Damit könn(t)en sich interessierte Forschende einen eigenen Überblick zu verwandten Arbeiten in
diesem Themengebiet schaffen, sofern dieser noch nicht vorhanden ist.
Es folgt die Auflistung ähnlicher bzw. verwandter Arbeiten.
Diese Arbeiten sollten dabei nicht einfach beschrieben werden, sondern immer in Bezug zur eigenen
Arbeit gesetzt werden.
Idealerweise sollten die Kernverbesserungen der eigenen Arbeit im Verhältnis zu jeder verwandten
Arbeit hier herausgearbeitet werden.
Auf keinen Fall sollten verwandte Arbeiten unreflektiert aufgelistet werden.

\subsection{Zusammenfassung und Ausblick}
In der Zusammenfassung fassen Sie die Resultate Ihrer Ausarbeitung zusammen. Hinterfragen Sie die
Resultate, die Sie im Rahmen Ihrer Ausarbeitung dargestellt haben. Setzen Sie die Resultate in einen
größeren Kontext. Geben Sie eine (eigene) Wertung über die Resultate.

\subsection{Referenzen/Literatur} \label{subsec:referenzen}

Hier listen Sie die Literatur auf, auf die Sie im Rahmen Ihrer Arbeit verweisen. Dieser Abschnitt
wird vom Tool \hologo{BibTeX} automatisch entsprechend den Templateregeln generiert. Erfassen Sie
Ihre Referenzen in der Datei \texttt{bibtex.bib} aus der Vorlage. \hologo{BibTeX}-Einträge für
wissenschaftliche Veröffentlichungen können Sie z.~B. von \url{https://dblp.org} (siehe auch
Abbildung \ref{abb:dblp})\footnote{Sollte DBLP Ihnen zwei \hologo{BibTeX}-Einträge ausgeben, so
  sollten Sie beide Einträge in das Literaturverzeichnis kopieren.} oder auch von
\url{https://scholar.google.de} exportieren. Weiter können Sie für die Verwaltung der
\hologo{BibTeX}-Einträge auch Software wie z.~B.
\emph{JabRef}\footnote{\url{https://www.jabref.org/}} verwenden.

\begin{figure}
    \centering
    \includegraphics[width=\textwidth]{dblp.png}
    \caption{\hologo{BibTeX} Referenzen aus DBLP exportieren.}
    \label{abb:dblp}
\end{figure}


\section{Formalia}
In diesem Abschnitt werden kurz die wichtigsten Eckpunkte zur Beachtung der Formalia einer
Seminararbeit dargestellt. Viele Formalia, wie etwa Schriftgröße und Schriftart, sind bereits durch
dieses (verpflichtende) Template vorgegeben und werden hier nicht separat erwähnt.


\subsection{Deckblatt}
Sie sollten folgende Regeln für das Erstellen des Deckblattes beachten:
\begin{itemize}
\item Geben Sie an, ob es sich um eine Seminarausarbeitung (Bachelor) oder eine
  Hauptseminarausarbeitung (Master) handelt und hinterlegen Sie die korrekte Veranstaltung.
\item Geben Sie Ihren Studiengang an (siehe auch Tabelle~\ref{tab:studiengangauswahl}).
\item Geben Sie Ihren Namen \textbf{und Matrikelnummer} an. Sofern Sie bereits einen akademischen
  Grad erreicht haben, geben Sie diesen mit an.
\end{itemize}

\begin{table}
    \begin{tabular}{l|l}
        \textbf{Eingabewert} & \textbf{Studiengang}\\
        \hline
        cs & Informatik\\
        se & Softwaretechnik\\
        mcl & Computerlinguistik\\ 
        tk & Technische Kybernetik\\
        msv & Maschinelle Sprachverarbeitung\\
        bis & Wirtschaftsinformatik\\
        simtech & Simulation Technology\\
        Studiengang & ``Studiengang'' (nicht vordefinierter Studiengang)\\ 
    \end{tabular}
    \caption{Deckblatt: Studiengangauswahl im Template im Befehl \texttt{\textbackslash course}}
    \label{tab:studiengangauswahl}
\end{table}


\subsection{Sprache, Orthografie und Grammatik}
Verfassen Sie die Ausarbeitung in Deutsch oder Englisch\footnote{Zu Beginn der \hologo{TeX}-Datei
  müssen Sie die Sprache entsprechend einstellen.}. Sowohl Orthografie als auch Grammatik fließen in
die Bewertung ein und sollten daher fehlerfrei sein.  Sollten Sie die Ausarbeitung in Englisch
verfassen, halten Sie sich an die in der Informatik übliche amerikanische Rechtschreibung.  Für die
meisten \LaTeX-fähigen Editoren (z.~B. TeXStudio) gibt es Erweiterungen zur Überprüfung der
Rechtschreibung -- wir empfehlen Ihnen, diese zu verwenden.

In einer deutschen Ausarbeitung verwenden Sie in der Informatik üblicherweise auch englische
Fachbegriffe -- sofern es keine etablierten deutschen Fachausdrücke als Gegenstück
gibt. Beispielsweise ist der Begriff ``Reihung'' den meisten Informatiker*innen nicht geläufig, während
jeder den entsprechenden englischen Begriff ``Array'' kennt.

\subsection{Zitieren}
Verweisen Sie im Rahmen Ihrer Ausarbeitung in den jeweiligen Textpassagen auf die Quellen, die Sie
verwendet haben bzw. deren Inhalt Sie darstellen.  Dabei sollten Ihre Literaturangaben das Auffinden
der Quellen möglichst einfach machen.  Beispiele für Zitierweisen:
\begin{itemize}
\item ... diese Resultate konnten bestätigt werden \cite{stickel2009wissenschaftliches}.
\item Wie Bohlinger \cite{BOH07} gezeigt hat, ...
\item Wie in \cite{KAR} dargestellt, ...
\item Der im folgenden besprochene Angriff wurde in \cite{DBLP:conf/ccs/AdrianBDGGHHSTV15} erstmalig
  beschrieben.
\end{itemize}
Vermeiden Sie durch geeignete Zitierweise, dass Sie in längeren Abschnitten jeden einzelnen Satz mit
der gleichen Referenz versehen.

\subsection{Tabellen, Grafiken und Code Fragmente}
Tabellen, Grafiken und Codefragmente sollten Sie jeweils mit einer Beschriftung versehen. Verweisen
Sie innerhalb Ihres Textes an der passenden Stelle auf das entsprechende Objekt. \hologo{LaTeX}
unterstützt Sie bei der korrekten Darstellung mit dem \texttt{table}, \texttt{graphicx} und
\texttt{listings}-Paket. Der Abschnitt~\ref{subsec:referenzen} gibt Ihnen ein Beispiel für die
Verwendung einer Grafik in der Ausarbeitung. (In wissenschaftlicher Literatur ist es üblich,
Grafiken nicht als Teil des Fließtextes einzubinden, sondern zu Beginn einer Seite oder auf einer
separaten Seite darzustellen -- die Platzierung übernimmt \hologo{LaTeX} für Sie.)

\section{Weitere Hinweise}
\begin{itemize}
\item Erstellen Sie Ihre Ausarbeitung \emph{vor} Ihrem Seminarvortrag und nutzen Sie die Zeit nach
  dem Vortrag zur Überarbeitung/Verbesserung der Ausarbeitung.
\item Bei Fragen und Unklarheiten, wenden Sie sich selbstständig an Ihre Betreuungsperson.
\item Ein guter Ausgangspunkt zum Finden von weiterführender Literatur sind unter anderem die
  Referenzen in ihrer Themenvorlage.
\item Stellen Sie sicher, dass Sie Konzepte einführen, bevor Sie diese verwenden. Dieser Hinweis
  gilt insbesondere auch für die Verwendung von Abkürzungen.
\item Nutzen Sie den Platz in der Ausarbeitung für die Darstellung von Inhalten sinnvoll
  aus. Vermeiden Sie Platzverschwendung (z.~B. durch schlecht skalierte Grafiken).
\item Überschriften sollten immer von Fließtext gefolgt sein, nicht unmittelbar von einer weiteren
  Unterüberschrift.
\item Eine Überschriftenebene sollten Sie nur mit Unterüberschriften unterteilen, wenn Sie
  mindestens zwei Unterabschnitte anlegen.
\item Fügen Sie \emph{keinen} manuellen Seitenumbruch nach Abschluss eines Abschnittes ein.
\item Hilfreich ist es, Ihre Arbeit Dritten, z.~B. Kommiliton*innen, zum Lesen zu geben. Sie können So
  wertvolles Feedback zum Inhalt und der Rechtschreibung Ihrer Ausarbeitung erhalten.
\end{itemize}


\bibliographystyle{plainnat}
\bibliography{bibtex}
\end{document}






%%% Local Variables:
%%% mode: latex
%%% TeX-master: t
%%% End:
