\section*{Guide ---------------------------------------------------}

\section{Ziele einer Seminararbeit}
Im Rahmen des Seminars ``Informationssicherheit und Kryptographie'' haben Sie die Aufgabe,
``Expert*in'' zu einem wissenschaftlichem Thema\footnote{Beachten Sie, dass Ihr Thema von Ihrem
Betreuer genauer eingegrenzt und Schwerpunkte definiert werden können.} zu werden. Dies bedeutet
insbesondere, dass
\begin{itemize}
\item Sie Ihr Thema inhaltlich verstehen,
\item Sie sich zu Ihrem Thema eigene Gedanken machen, und
\item Sie sich auch über die vorgegeben Unterlagen hinaus über Ihr Thema informieren.
\end{itemize}


Das Ziel der Seminararbeit ist, den Lesenden dieses fachliche Thema näher zu bringen und zu
erläutern. Folgende Punkte sind dabei insbesondere zu beachten:

\begin{itemize}
\item Ihre Aufgabe besteht nicht ausschließlich in der reinen Wiedergabe eines Themas. Auch die
  Ausarbeitung sollte zeigen, dass Sie sich eigene Gedanken zu Ihrem Thema gemacht haben.
\item Die Zielgruppe Ihrer Ausarbeitung sind Kommiliton*innen mit ähnlichem Wissensstand wie Sie. Das
  heißt, Sie können erwarten, dass Lesende Einführungsveranstaltungen im Bereich Informatik
  (einschließlich der Veranstaltung \emph{Grundlagen der Informationssicherheit}) erfolgreich
  besucht haben.
\item Ihre Ausarbeitung sollte in sich geschlossen sein.  Lesende müssen von Ihnen in die Lage
  versetzt werden, Ihre Ausarbeitung ohne Zuhilfenahme weiterer Literatur zu verstehen. Daher:
  Führen Sie grundlegende Begriffe und Konzepte ein, falls Sie nicht erwarten können, dass Lesende
  mit ihnen vertraut sind. Sollten Sie sich bei einem Aspekt nicht sicher sein, ob Sie deren Kenntnis
  bei den Lesenden voraussetzen können, fragen Sie Ihre Betreuungsperson.
\item Üblicherweise können Sie viel mehr zu einem Thema schreiben, als Sie Platz zur Verfügung
  haben. Es ist Teil Ihrer Aufgabe, das Thema in dem vorgegebenen Rahmen möglichst gut
  darzustellen. Dazu müssen Sie ggf. an einigen Stellen abstrahieren.
\item Vermitteln Sie neben den technischen Fakten und Sachverhalten den Lesenden auch eine Intuition
  für Ihr Thema. Veranschaulichungen, Beispiele und Bilder können --~in Maßen verwendet~-- Lesende
  beim Verstehen Ihres Themas unterstützen.  Finden Sie ein gutes Mittelmaß zwischen oberflächlicher
  Betrachtung und technischen Details.
\item Englische Fachbegriffe sollten nur sehr vorsichtig (oder besser gar nicht) eingedeutscht
  werden. Im Zweifel lieber zu wenig als zu viel eindeutschen bzw. gleich die ganze Ausarbeitung in
  englischer Sprache verfassen.
\end{itemize}

\section{Struktur der Ausarbeitung}
Eine wissenschaftliche Ausarbeitung sollte sinnvoll strukturiert sein. Das heißt, es sollte einen
roten Faden geben, der Lesende sinnvoll durch das Thema führt. Stellen Sie sicher, dass den Lesenden
zu jedem Zeitpunkt der Sinn und Zweck des aktuellen Kapitels klar ist, etwa indem Sie vorher einen
kurzen Überblick geben.

Im Folgenden zeigen wir Ihnen beispielhaft eine übliche Struktur für wissenschaftliche
Ausarbeitungen. Dabei sind Struktur, Reihenfolge, als auch Benennung der Abschnitte \emph{nicht
  strikt festgelegt}. Überlegen und entscheiden Sie selbst, wie Sie Ihr Thema verständlich und
technisch korrekt darstellen können.

\subsection{Kurzfassung / Abstract}
In der Kurzfassung (englisch: Abstract) geben Sie eine möglichst kurze und prägnante Inhaltsangabe
Ihrer Arbeit.
Sie sollte alle wichtigen Schlüsselwörter enthalten und eine ähnliche Struktur wie die Arbeit selber
haben.
Als Faustregel sollte jeder Abschnitt (``section'') in ca. einem Satz zusammengefasst werden.
Eine typische Länge für die Kurzfassung sind 150 bis 200 Wörter.

\subsection{Einleitung}
Eine Einleitung führt Lesende in das Thema der Ausarbeitung bzw. einer wissenschaftlichen Arbeit
ein. Erklären Sie, welches Problem Sie in Ihrer Ausarbeitung behandeln und warum dieses Problem ein
relevantes Problem ist. Vielleicht setzten Sie ihr Thema in einen sinnvollen Kontext. Geben Sie
einen Über- und Ausblick über die Ausarbeitung. Welcher Abschnitt behandelt welches Thema?

\subsection{Verwandte Arbeiten und Wissenschaftlicher Hintergrund}

Eine Auflistung verwandter Arbeiten ist essentiell für jeden wissenschaftlichen Artikel.\\

\noindent
\emph{Zur Position.}
Bereits in der Einleitung sollte auf die Unzulänglichkeiten verwandter Arbeiten eingegangen werden,
um die eigenen Arbeit zu motivieren.
In kürzeren Ausarbeitung mit einer kleinen Anzahl ähnlicher veröffentlichter wissenschaftlicher
Arbeiten kann man die verwandten Arbeiten alternativ schon komplett in der Einleitung abhandeln.
In längeren Ausarbeitungen oder sobald der Umfang der verwandten Literatur einen eigenen Abschnitt
rechtfertigt, sollte jedoch ein eigener Abschnitt "Related Work/Verwandte Arbeiten" angelegt werden.
In jedem Fall sollte das Thema "Verwandte Arbeiten" für Lesende erkennbar und nachvollziehar
dargestellt werden.
Der eigene Abschnitt kann entweder vor den eigentlichen Beitrag gestellt werden (also der Abschnitt
nach der Einleitung), um die Verbesserungen der Arbeit gegenüber anderen Arbeiten zu betonen.
Alternativ kann dieser Abschnitt auch erst am Ende eines Papers vor Zusammenfassung und Ausblick
auftauchen.\\

\noindent
\emph{Zum Inhalt}.
In den Abschnitt "Related Work" gehört beispielsweise ein bekannter Überblicks-Artikel zum
Themengebiet bzw. ein standard-referenziertes Buch.
Damit könn(t)en sich interessierte Forschende einen eigenen Überblick zu verwandten Arbeiten in
diesem Themengebiet schaffen, sofern dieser noch nicht vorhanden ist.
Es folgt die Auflistung ähnlicher bzw. verwandter Arbeiten.
Diese Arbeiten sollten dabei nicht einfach beschrieben werden, sondern immer in Bezug zur eigenen
Arbeit gesetzt werden.
Idealerweise sollten die Kernverbesserungen der eigenen Arbeit im Verhältnis zu jeder verwandten
Arbeit hier herausgearbeitet werden.
Auf keinen Fall sollten verwandte Arbeiten unreflektiert aufgelistet werden.

\subsection{Zusammenfassung und Ausblick}
In der Zusammenfassung fassen Sie die Resultate Ihrer Ausarbeitung zusammen. Hinterfragen Sie die
Resultate, die Sie im Rahmen Ihrer Ausarbeitung dargestellt haben. Setzen Sie die Resultate in einen
größeren Kontext. Geben Sie eine (eigene) Wertung über die Resultate.



\subsection{Grundlagen}
Führen Sie alle Begriffe, die Ihre Lesergruppe potentiell nicht kennt, ein. Wählen Sie dabei die
passende Detailtiefe: Müssen Sie einen Begriff formal definieren oder reicht es, den Lesenden eine
Intuition zu vermitteln? Beschreiben Sie die Grundlagen prägnant.

\subsection{Hauptteil}
Der Hauptteil der Ausarbeitung ist der eigentliche Kern Ihrer Arbeit, in dem Sie ihr Thema den Lesenden
erklären. Der Begriff ``Hauptteil'' bezeichnet hierbei ein Konzept und ist keinesfalls die
Überschrift für diesen Abschnitt. Normalerweise umfasst der Hauptteil ohnehin mehrere Kapitel.

\section{Formalia}

\subsection{Zitieren}
\begin{itemize}
\item ... diese Resultate konnten bestätigt werden \cite{stickel2009wissenschaftliches}.
\item Wie Bohlinger \cite{BOH07} gezeigt hat, ...
\end{itemize}

\section{Weitere Hinweise}
\begin{itemize}
\item Erstellen Sie Ihre Ausarbeitung \emph{vor} Ihrem Seminarvortrag und nutzen Sie die Zeit nach
  dem Vortrag zur Überarbeitung/Verbesserung der Ausarbeitung.
\item Stellen Sie sicher, dass Sie Konzepte einführen, bevor Sie diese verwenden. Dieser Hinweis
  gilt insbesondere auch für die Verwendung von Abkürzungen.
\item Überschriften sollten immer von Fließtext gefolgt sein, nicht unmittelbar von einer weiteren
  Unterüberschrift.
\item Eine Überschriftenebene sollten Sie nur mit Unterüberschriften unterteilen, wenn Sie
  mindestens zwei Unterabschnitte anlegen.
\end{itemize}
