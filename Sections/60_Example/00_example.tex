\section{The Example of $S_3$}

In this section, we will create the character table for the symmetric group $S_3$.
We start by describing $S_3$.
Then we will find all inequivalent irreducible representations and list their characters in a character table.

\subsection{The Group $S_3$}

The group $S_3$ consists of all possible permutations of $3$ objects.
We can think of each element as an action on a triangle as shown in \Cref{fig:reprep.char.triangle}.
Elements of the group are usually written as numbers ($1-3$) in brackets, such as $(123)$.
The notation will tell us, which corner is mapped where.
The corner with the first number will be mapped to the corner with the second number.
If there is a third number, the corner with the second number will be mapped to the corner with the third number.
And finally, the corner with the last number will be mapped to the corner with the first number.
So the group element (12) will just swap corners 1 and 2 and the group element (123) will rotate the triangle clockwise.

The group has 3 conjugacy classes.
The identity $(1)$ is its own conjugacy class, since for any $g \in S_3$ the following holds.
\begin{align}
    g \cdot (1) \cdot g^{-1} & = g \cdot g^{-1} = (1)
\end{align}
It has order 1.
Another conjugacy class is the transpositions $\{(12), (23), (13)\}$. 
Each element in this class is its own inverse, so its order is 2.
And the last conjugacy class is the rotations $\{(123), (132)\}$.
Since it takes 3 rotations to rotate the triangle by 360\textdegree, their order is 3.

\begin{figure}[!h]
    \centering
    
    \begin{tikzpicture}
    \draw   (0, 0) node [anchor=north east] {1}
        --  (60:2) node [anchor=south] {2}
        --  (2, 0) node [anchor=north west] {3}
        --  cycle;
    \end{tikzpicture}

    \caption{Triangle for visualizing $S_3$}
    \label{fig:reprep.char.triangle}
\end{figure}

\subsection{The Trivial Representation and The Sign Representation}

The easiest representation of every finite group is the trivial representation $\triv$.
We know, that it maps every group element $a \in S_3$ to the identity map of the vector space $\C^1$.
So $\triv$ is simply $\triv = 1$.
Its character therefore is $\chi_{\triv} = 1$.
This representation is irreducible because the only subspaces of a one-dimensional vector space $V$ are $V$ itself and $\{0_V\}$~\cite{fulton2013}.

The sign representation $\sgn$ is a little bit more complicated.
Suppose, we want another one-dimensional representation.
Since for one-dimensional representations the character is identical to the representation, we can assume that the representation is constant on the conjugacy classes of $S_3$.
The identity element $(1)$ must always be represented as the identity map of the target vector space.
So $\sgn((1)) = 1$ for our one-dimensional case.
Since the elements in the conjugacy class $a \in \Cl((12))$ have order 2, we now choose to map them to $\sgn(a) = -1$ instead of $1$.
Then for any $a \in \Cl((12))$ the representation of $a \cdot a$ is $\sgn(a \cdot a) = -1 \cdot -1 = 1 = \sgn((1))$.
This reassures us, that the choice of $\sgn((a)) = -1$ is valid for the elements $a \in \Cl((12))$ because $\sgn$ acts as a homomorphism on these elements and the identity element.
Now we just need to find the representation of the elements of the conjugacy class $a \in \Cl((123))$.
We can construct the element $(123)$ from transpositions $(123) = (12) \cdot (23)$.
Since $\sgn$ is a homomorphism, we now can calculate the representation of $(123)$ and therefore of all elements in its conjugacy class $a \in \Cl((123))$.
\begin{align}
    \sgn((123)) & = \sgn((12) \cdot (23)) = \sgn((12)) \cdot \sgn((23))  = -1 \cdot -1 = 1
\end{align}
So in summary,
\begin{align}
    \sgn(a) & = \begin{cases}
        -1 & \text{ if } a \in \Cl((12)) \\
        1 & \text{ else}
    \end{cases}
\end{align}
Similarily the character is
\begin{align}
    \chi_{\sgn}(a) & = \begin{cases}
        -1 & \text{ if } a \in \Cl((12)) \\
        1 & \text{ else}
    \end{cases}
\end{align}

\subsection{The Standard Representation}

Assume, that every corner of the triangle in \Cref{fig:reprep.char.triangle} is associated with a value $x_i$ where $x_1 + x_2 + x_3 = 0$.
This subspace of $\C^3$ can be thought of a 2-dimensional vector space with the basis $\{e_1 - e_2, e_2 - e_3\}$ with
\begin{align*}
    x' = \begin{pmatrix}
        x_1 - x_2 \\
        x_2 - x_3
    \end{pmatrix}
\end{align*}
The elements of $S_3$ map the values of $x_i$ between the corners as described above.
So the standard representation maps 1 to I because all $x_i$s stay the same and therefore also $x'_1 = x_1 - x_2$ and $x_2' = x_2 - x_3$.
I will write $\stand((12))$ as $s^{(12)}$ in the following for readability.
The action (12) swaps the values of $x_1$ and $x_2$.
So $s^{(12)}_{00} = -1$ and $s^{(12)}_{01} = 0$, since $x'_1$ is just negated.
For $x'_2$ it is a little more complicated, the elements of the matrix must satisfy the equality
\begin{align*}
    x_1 - x_3 & = s^{(12)}_{10} \cdot (x_1 - x_2) + s^{(12)}_{11} \cdot (x_2 - x_3) \\
    & = x_1 \cdot s^{(12)}_{10} + x_2 \cdot (s^{(12)}_{11} - s^{(12)}_{10}) + x_3 \cdot s^{(12)}_{11} \\
\end{align*}
So $s^{(12)}_{10} = s^{(12)}_{11} = 1$ and the resulting matrix is
\begin{align*}
    s^{(12)} & = \begin{pmatrix}
        -1 & 0 \\
        1 & 1
    \end{pmatrix}
\end{align*}
Its trace is 0.
We don't need to compute the matrices for the other two elements in the same conjugacy class, since the character is constant on the conjugacy classes as mentioned above.
Now we computed the values of the character of the standard representation of $S_3$ for the conjugacy classes of 1 and (12).
The value for the conjugacy class (123) is missing.
I will write $\stand((123))$ as $s^{(123)}$ in the following for readability.
For to get the correct value in $x'_1$, the elements of the matrix must satisfy the following equality
\begin{align*}
    x_2 - x_3 & = s^{(123)}_{00} \cdot (x_1 - x_2) + s^{(123)}_{01} \cdot (x_2 - x_3)
\end{align*}
So $s^{(123)}_{00} = 0$ and $s^{(123)}_{01} = 1$.
For $x'_2$ it is more complicated and the remaining elements of the matrix have to satisfy the equality
\begin{align*}
    x_3 - x_1 & = s^{(123)}_{10} \cdot (x_1 - x_2) + s^{(123)}_{11} \cdot (x_2 - x_3) \\
    & = x_1 \cdot s^{(123)}_{10} + x_2 \cdot (s^{(123)}_{11} - s^{(123)}_{10}) + x_3 \cdot s^{(123)}_{11}
\end{align*}
So $s^{(123)}_{10} = s^{(123)}_{11} = -1$ and  the resulting matrix with trace -1 is
\begin{align*}
    s^{(123)} & = \begin{pmatrix}
        0 & 1 \\
        -1 & -1
    \end{pmatrix}
\end{align*}

\todo[inline]{proof faithful}
\todo[inline]{proof irreducible}

\subsection{The Character Table}

\todo[inline]{full table description etc}

\begin{table}
    \centering

    \begin{tabular}{r | c  c  c}
                    & 1 & 3     & 2     \\
        $S_3$       & 1 & (12)  & (123) \\ \hline
        $\triv$     & 1 & 1     & 1     \\
        $\sgn$      & 1 & -1    & 1     \\
        $\stand$    & 2 & 0     & -1
    \end{tabular}

    \caption{Character table for the group $S_3$}
    \label{tab:reprep.char.table}
\end{table}

\Cref{tab:reprep.char.table} is an example of a character table.
It shows all simple representations of the group $S_3$.
$S_3$ is the group of symmetries of a triangle.

\subsection{General Standard Representation and Meaning for Cryptography}
