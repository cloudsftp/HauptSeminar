\section{The Example of $S_3$}

In this section, we will create the character table for the symmetric group $S_3$.
We start by describing $S_3$.
Then we will find all inequivalent irreducible representations and list their characters in a character table.

\subsection{The Group $S_3$}
\label{sec:ex.group}

The group $S_3$ consists of all possible permutations of $3$ objects.
We can think of each element as an action on a triangle as shown in \Cref{fig:reprep.char.triangle}.
Elements of the group are usually written as numbers ($1-3$) in brackets, such as $(123)$.
The notation will tell us, which corner is mapped where.
The corner with the first number will be mapped to the corner with the second number.
If there is a third number, the corner with the second number will be mapped to the corner with the third number.
And finally, the corner with the last number will be mapped to the corner with the first number.
So the group element (12) will just swap corners 1 and 2 and the group element (123) will rotate the triangle clockwise.

The group has 3 conjugacy classes.
The identity $(1)$ is its own conjugacy class, since for any $g \in S_3$ the following holds.
\begin{align}
    g \cdot (1) \cdot g^{-1} & = g \cdot g^{-1} = (1)
\end{align}
It has order 1.
Another conjugacy class is the transpositions $\{(12), (23), (13)\}$. 
Each element in this class is its own inverse, so its order is 2.
And the last conjugacy class is the rotations $\{(123), (132)\}$.
Since it takes 3 rotations to rotate the triangle by 360\textdegree, their order is 3.

\begin{figure}[!h]
    \centering
    
    \begin{tikzpicture}
    \draw   (0, 0) node [anchor=north east] {1}
        --  (60:2) node [anchor=south] {2}
        --  (2, 0) node [anchor=north west] {3}
        --  cycle;
    \end{tikzpicture}

    \caption{Triangle for visualizing $S_3$}
    \label{fig:reprep.char.triangle}
\end{figure}

\subsection{The Trivial Representation and The Sign Representation}

The easiest representation of every finite group is the trivial representation $\triv$.
We know, that it maps every group element $a \in S_3$ to the identity map of the vector space $\C^1$.
So $\triv$ is simply $\triv = 1$.
Its character therefore is $\chi_{\triv} = 1$.
This representation is irreducible because the only subspaces of a one-dimensional vector space $V$ are $V$ itself and $\{0_V\}$~\cite{fulton2013}.

The sign representation $\sgn$ is a little bit more complicated.
Suppose, we want another one-dimensional representation.
Since for one-dimensional representations, the character is identical to the representation, we can assume that the representation is constant on the conjugacy classes of $S_3$.
The identity element $(1)$ must always be represented as the identity map of the target vector space.
So $\sgn((1)) = 1$ for our one-dimensional case.
Since the elements in the conjugacy class $a \in \Cl((12))$ have order 2, we now choose to map them to $\sgn(a) = -1$ instead of $1$.
Then for any $a \in \Cl((12))$ the representation of $a \cdot a$ is $\sgn(a \cdot a) = -1 \cdot -1 = 1 = \sgn((1))$.
This reassures us, that the choice of $\sgn((a)) = -1$ is valid for the elements $a \in \Cl((12))$ because $\sgn$ acts as a homomorphism on these elements and the identity element.
Now we just need to find the representation of the elements of the conjugacy class $a \in \Cl((123))$.
We can construct the element $(123)$ from transpositions $(123) = (12) \cdot (23)$.
Since $\sgn$ is a homomorphism, we now can calculate the representation of $(123)$ and therefore of all elements in its conjugacy class $a \in \Cl((123))$.
\begin{align}
    \sgn((123)) & = \sgn((12) \cdot (23)) = \sgn((12)) \cdot \sgn((23))  = -1 \cdot -1 = 1
\end{align}
So in summary,
\begin{align}
    \sgn(a) & = \begin{cases}
        -1 & \text{ if } a \in \Cl((12)) \\
        1 & \text{ else}
    \end{cases}
\end{align}
Similarly the character is
\begin{align}
    \chi_{\sgn}(a) & = \begin{cases}
        -1 & \text{ if } a \in \Cl((12)) \\
        1 & \text{ else}
    \end{cases}
\end{align}
As for the trivial representation, this representation is also irreducible for the same reason~\cite{fulton2013}.

\subsection{The Standard Representation}
\label{sec:ex.stand}

The standard representation is our first and only irreducible faithful representation.
Assume, that every corner of the triangle in \Cref{fig:reprep.char.triangle} is associated with a coordinate of $\C^3$, $x_i$.
When permutating these values, the sum stays constant.
Without loss of generality, we can assume that the sum is $x_1 + x_2 + x_3 = 0$.
This is a subspace $V' = \{x \in \C^3 | x_1 + x_2 + x_3 = 0\}$ of $\C^3$, which is equivalent to $\C^2$ by the isomorphism
\begin{align}
    \pi: \qquad & V' \to \C^2, \quad \begin{pmatrix}
        x_1 \\ x_2 \\ x_3
    \end{pmatrix} \mapsto \begin{pmatrix}
        x_1 - x_2 \\
        x_2 - x_3
    \end{pmatrix}
\end{align}

The elements of $S_3$ permutate the values $x_i$ as described in \Cref{sec:ex.group}.
The image if $(1)$ is $I$, like for every representation.
The image of the other elements has to be computed by its action on the vectors of $\C^2$, which can be derived from its actions on $V'$.
For example, take the element $(12)$.
Its action $\rho': S_3 \to \GL(V')$ is
\begin{align}
    \begin{pmatrix}
        x_2 \\ x_1 \\ x_3
    \end{pmatrix} = \rho'((12)) \cdot \begin{pmatrix}
        x_1 \\ x_2 \\ x_3
    \end{pmatrix}
\end{align}
Therefore, its action $\stand: G \to \GL(\C^2)$ on $\C^2$ must be
\begin{align}
    \begin{pmatrix}
        x_2 - x_1 \\
        x_1 - x_3
    \end{pmatrix} = \stand((12)) \cdot \begin{pmatrix}
        x_1 - x_2 \\
        x_2 - x_3
    \end{pmatrix}
\end{align}
We can calculate the matrix $\stand((12))$ via two linear equations.
First, we calculate the elements in the top row.
We will write them as $s^{(12)}_{11}$ and $s^{(12)}_{12}$.
\begin{subequations}
\begin{align}
    x_2 - x_1 & = s^{(12)}_{11} \cdot (x_1 - x_2) + s^{(12)}_{12} \cdot (x_2 - x_3) \\
    & = (-1) \cdot (x_1 - x_2) \\
    \Longrightarrow \quad & s^{(12)}_{11} = -1 \land s^{(12)}_{12} = 0
\end{align}
\end{subequations}
Similarly, we can get the elements in the bottom row, $s^{(12)}_{21}$ and $s^{(12)}_{22}$.
\begin{subequations}
\begin{align}
    x_1 - x_3 & = s^{(12)}_{21} \cdot (x_1 - x_2) + s^{(12)}_{22} \cdot (x_2 - x_3) \\
    & = x_1 - x_2 + x_2 - x_3 \\
    \Longrightarrow \quad & s^{(12)}_{21} = 1 \land s^{(12)}_{22} = 1
\end{align}
\end{subequations}

We can do this for every element in $S_3$ and will get the following matrices.
\begin{table}[h]
    \centering
    \begin{tabular}{c c c}
        $\stand((1)) = I$ & $\stand((123)) = \begin{pmatrix}
            -1 & -1 \\
            1 & 0
        \end{pmatrix}$ & $\stand((132)) = \begin{pmatrix}
            0 & 1 \\
            -1 & -1
        \end{pmatrix}$ \\\\ $\stand((12)) = \begin{pmatrix}
            -1 & 0 \\
            1 & 1
        \end{pmatrix}$ & $\stand((13)) = \begin{pmatrix}
            0 & -1 \\
            -1 & 0
        \end{pmatrix}$ & $\stand((23)) = \begin{pmatrix}
            1 & 1 \\
            0 & -1
        \end{pmatrix}$
    \end{tabular}
\end{table}
As we can see, all matrices are different from each other.
This means, that $\stand$ is faithful since every element $a \in S_3$ has a different image under $\stand$.

We now want to prove that $\stand$ is irreducible.
For it to be irreducible, there may be no non-trivial subspace, that is invariant under $\stand$.
We start by identifying vector subspaces of $\C^2$ that are invariant under $\stand((13))$, they are our candidates.
These subspaces are the eigenspaces of $\stand((13))$.
First, we calculate the eigenvalues:
\begin{subequations}
\begin{align}
    \det \left[ \stand((13)) - \lambda \cdot I \right]
    & = \det \left[ \begin{pmatrix}
        0 & -1 \\ -1 & 0
    \end{pmatrix} - \lambda \cdot I \right] = \det \left[ \begin{pmatrix}
        -\lambda & -1 \\ -1 & -\lambda
    \end{pmatrix} \right] \\
    & = \lambda^2 - 1 = 0
\end{align}
\end{subequations}
Therefore the eigenvalues are $\lambda_1 = 1$ and $\lambda_2 = -1$.
The corresponding eigenvectors are $v_1 = \begin{pmatrix}
    1 & -1
\end{pmatrix}^T$ and $v_2 = \begin{pmatrix}
    1 & 1
\end{pmatrix}^T$.
Therefore the eigenspaces are $E_{\lambda_1} = \C \cdot v_1$ and $E_{\lambda_2} = \C \cdot v_2$.
Now, we show that both these vector subspaces are not invariant under $\stand((123))$:
\begin{subequations}
\begin{align}
    \stand((123)) \cdot v_1 & = \begin{pmatrix}
        -1 & -1 \\
        1 & 0
    \end{pmatrix} \cdot \begin{pmatrix}
        1 \\ -1
    \end{pmatrix} = \begin{pmatrix}
        0 \\ 1
    \end{pmatrix} \notin E_{\lambda_1} \\
    \stand((123)) \cdot v_2 & = \begin{pmatrix}
        -1 & -1 \\
        1 & 0
    \end{pmatrix} \cdot \begin{pmatrix}
        1 \\ 1
    \end{pmatrix} = \begin{pmatrix}
        -2 \\ 1
    \end{pmatrix} \notin E_{\lambda_2}
\end{align}
\end{subequations}
Therefore, there can be no non-trivial vector subspace that is invariant under $\stand$~\cite{fulton2013}.

The character is calculated quickly by choosing one element from each conjugacy class $(1) \in \Cl((1))$, $(12) \in \Cl((12))$, and $(123) \in \Cl((123))$, and then taking the trace of its image in $\stand$.
\begin{subequations}
\begin{align}
    \chi_{\stand}((1)) & = \Tr\left( \begin{pmatrix}
        1 & 0 \\
        0 & 1
    \end{pmatrix} \right) = 2 \\
    \chi_{\stand}((12)) & = \Tr\left( \begin{pmatrix}
        -1 & 0 \\
        1 & 1
    \end{pmatrix} \right) = 0 \\
    \chi_{\stand}((123)) & = \Tr\left( \begin{pmatrix}
        -1 & -1 \\
        1 & 0
    \end{pmatrix} \right) = -1
\end{align}
\end{subequations}
Therefore the character is given by the following equation~\cite{fulton2013}.
\begin{align}
    \chi_{\stand}(a) = \begin{cases}
        2 & \text{ if } a \in \Cl((1)) \\
        0 & \text{ if } a \in \Cl((12)) \\
        -1 & \text{ if } a \in \Cl((123))
    \end{cases}
\end{align}

\subsection{The Character Table}

\Cref{tab:reprep.char.table} lists the characters of all the inequivalent irreducible representations we found of the group $S_3$.
We can see, that it has as many rows as columns, therefore we know that we found all inequivalent irreducible representations.
We proved this fact in \Cref{sec:reprep.char}.

\begin{table}[h]
    \centering

    \begin{tabular}{r | c  c  c}
                    & 1 & 3     & 2     \\
        $S_3$       & 1 & (12)  & (123) \\ \hline
        $\triv$     & 1 & 1     & 1     \\
        $\sgn$      & 1 & -1    & 1     \\
        $\stand$    & 2 & 0     & -1
    \end{tabular}

    \caption{Character table for the group $S_3$}
    \label{tab:reprep.char.table}
\end{table}

\subsection{General Standard Representation and Meaning for Cryptography}

The standard representation, we constructed in \Cref{sec:ex.stand} can be constructed for any symmetric group $S_n$.
The construction works analogously.
This representation always has the dimension $n - 1$, but the group $S_n$ has $n!$ elements.
Since this representation is small compared to the size of the group, the group $S_n$ can't be used as a platform for most key-exchange protocols that work with non-abelian groups.
Such as the Shpilrain-Ushakov key-exchange~\cite{khovanov2022monoidal}.
