\subsection{Character Tables}

Character tables are the compact way, to write down representations, which was promised at the beginning of the section.
But first, we have to define what \textbf{characters} are.
Suppose $\rho: G \to \GL(V)$ is a representation of a finite group $G$.
The character of $\rho$ is defined as the map
\begin{subequations}
    \begin{align}
        \chi_\rho: \qquad & G \to \C, \\
        & a \mapsto \Tr(\rho(a))
    \end{align}
\end{subequations}
\todo[inline]{character equal for equivalent rep}
A representation can be recovered completely from its character.
\todo[inline]{proof?}
So now we know, we only have to write down one number per element $g \in G$.
\todo[inline]{diff proof}
But since the trace of a matrix stays the same after a base change, it is also true that
\begin{align}
    \forall g, h \in G: \qquad & \chi_\rho(h \cdot g \cdot h^{-1}) = \chi_\rho(g)
\end{align}
This tells us that the character $\chi_\rho$ is constant on the conjugacy classes of $G$.
So we only have to write down one number per conjugacy class of $G$.
This saves a lot of space and contains all the information needed to recover the representation~\cite{fulton2013}.

\todo[inline]{proof about \# of irred representations}

\textbf{Character tables} are just a way to write down characters of representations of groups.
All representations in one table are representations of the same group.
Typically, all \textit{simple} representations of the group are listed in the table.
Each row corresponds to one representation.
Each column corresponds to one conjugacy class of the group.
The classes are written on top of every column and above that, the number of group elements of that conjugacy class is written.
The number of conjugacy classes and the number of simple representations is always the same, so the portion of the table showing the characters is always square.
In the top left corner, the name of the group is written~\cite{fulton2013}.
