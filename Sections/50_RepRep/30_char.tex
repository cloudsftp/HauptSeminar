\subsection{Character Tables}
\label{sec:reprep.char}

Character tables are the compact way, to write down representations, which was promised at the beginning of the section.
But first, we have to define what \textbf{characters} are.
Suppose $\rho: G \to \GL(V)$ is a representation of a finite group $G$.
The character of $\rho$ is defined as the map
\begin{subequations}
    \begin{align}
        \chi_\rho: \qquad & G \to \C, \\
        & a \mapsto \Tr(\rho(a))
    \end{align}
\end{subequations}
This character of a representation is identical to the character of equivalent representations.
To prove this, assume that we have two equivalent representations, $\rho: G \to V$ and $\rho': G \to V'$ with an intertwiner $f: V \to V'$ that is invertible.
Then the following holds for any $a \in G$.
\begin{subequations}
\begin{align}
    \chi_{\rho'}(a) & = \Tr(\rho'(a)) = \Tr(f \cdot \rho'(a) \cdot f^{-1}) \\
    & = \Tr((f \cdot \rho'(a)) \cdot f^{-1}) = \Tr(f^{-1} \cdot (f \cdot \rho'(a))) \\
    & = \Tr(f^{-1} \cdot f \cdot \rho'(a) )= \Tr(\rho(a)) = \chi_\rho(a)
\end{align}
\end{subequations}

A representation can be recovered completely from its character.
Of course, the representation might not exactly be the original representation, since the character of two equivalent representations is the same.
But the recovered representation will be equivalent to the original representation.

So now we know, we only have to write down one number per element $g \in G$.
But actually, the character of a representation is constant on the group's conjugacy classes, so we only need to write down one number per conjugacy class.
This is since for any $a, g \in G$ the following is true.
\begin{subequations}
\begin{align}
    \chi_\rho(g \cdot a \cdot g^{-1}) & = \Tr(\rho(g \cdot a \cdot g^{-1})) = \Tr(\rho(g) \cdot \rho(a) \cdot \rho(g)^{-1}) \\
    & = \Tr((\rho(g) \cdot \rho(a)) \cdot \rho(g)^{-1}) = \Tr(\rho(g)^{-1} \cdot (\rho(g) \cdot \rho(a))) \\
    & = \Tr(\rho(g)^{-1} \cdot \rho(g) \cdot \rho(a)) = \Tr(\rho(a)) = \chi_\rho(a) \\
\end{align}
\end{subequations}
This saves a lot of space and contains all the information needed to recover the representation~\cite{fulton2013}.

The character of the direct sum of two representations is the sum of their characters.
This is due to the property of the trace, that $\Tr(A \oplus B) = \Tr(A) + \Tr(B)$~\cite{fulton2013}.
\begin{subequations}
\begin{align}
    \chi_{\rho \oplus \rho'}(a) & = \Tr((\rho \oplus \rho')(a)) = \Tr(\rho(a) \oplus \rho'(a)) \\
    & = \Tr(\rho) + \Tr(\rho'(a)) = \chi_\rho(a) + \chi_{\rho'}(a)
\end{align}
\end{subequations}

The character of the inverse of some group element $a \in G$ is the complex conjugate of the character of the element $\chi_\rho(a^{-1}) = \overline{\chi_\rho(a)}$.
For the proof, we remember, that the trace of a matrix is the sum of its eigenvalues.
\begin{align}
    \chi_\rho(a) & = \Tr(\rho(a)) = \sum_i \lambda_i
\end{align}
We know, $a$ is of finite order, so $\rho(a)$ must also be of finite order.
Therefore, all eigenvalues $\lambda_i$ of $\rho(a)$ are roots of unity and on the unit circle.
The eigenvalues of the inverse matrix $\rho(a)^{-1}$ are the element-wise inverses of the eigenvalues $\lambda_i$ of the matrix $\rho(a)$.
Since all $\lambda_i$ are on the unit cycle, their inverses are their complex conjugate $\lambda_i^{-1} = \overline{\lambda_i}$.
The complex conjugate is additive and therefore the following holds~\cite{fulton2013}.
\begin{subequations}
\begin{align}
    \chi_\rho(a^{-1}) & = \Tr(\rho(a^{-1})) = \Tr(\rho(a)^{-1}) \\
    & = \sum_i \lambda_i^{-1} = \sum_i \overline{\lambda_i} = \overline{\sum_i \lambda_i} \\
    & = \overline{\Tr(\rho(a))} = \overline{\chi_\rho(a)}
\end{align}
\end{subequations}

We can take the scalar product of two characters.
It is defined as
\begin{align}
    \langle \chi_\rho, \chi_{\rho'} \rangle & = \dfrac{1}{|G|} \sum_{a \in G} \chi_\rho(a) \cdot \overline{\chi_{\rho'}(a)}
\end{align}
If for two representations the scalar product is $\langle \chi_\rho, \chi_{\rho'} \rangle = 0$, they are called orthogonal.
The notion is similar to vectors, two orthogonal characters are linearly independent and so are their representations.
Therefore all characters of a representation must be orthogonal to each other and there are at most so many characters of irreducible representations as there are conjugacy classes of the group.
The number of inequivalent irreducible representations is also limited to that number~\cite{fulton2013}.

Now, since we have an upper limit on the number of inequivalent irreducible representations, we need a lower limit.
Using characters, we can prove that there are at least as many inequivalent irreducible representations of a group $G$, as there are conjugacy classes of $G$.
For the proof, we introduce a class function $\alpha: G \to \C$, for which $\langle \alpha, \chi_\rho \rangle = 0$ for all irreducible representations $\rho: G \to V$.
If we can show, that $\alpha$ must be $\alpha = 0$, we showed that there are at least as many orthogonal characters as there are conjugacy classes.
Because if $\alpha$ is not necessarily $0$, then there exists a non-zero class function, that is orthogonal to all characters, so the number of characters is smaller than the number of conjugacy classes.
This means that there are at least as many inequivalent irreducible representations as there are conjugacy classes.
We construct $\phi_{\alpha, \rho} = \sum_{g \in G} \alpha(g) \cdot \rho(g^{-1})$.
Since this is a $G$-linear map of $V$ onto itself, via Shur's Lemma $\phi_{\alpha, \rho} = \lambda \cdot I$.
And its trace is
\begin{subequations}
\begin{align}
    \Tr(\phi_{\alpha, \rho}) & = \sum_{g \in G} \alpha(g) \cdot \chi_\rho(g^{-1}) \\
    & = \dfrac{|G|}{|G|} \cdot \sum_{g \in G} \alpha(g) \cdot \chi_\rho(g^{-1})
    = \dfrac{|G|}{|G|} \cdot \sum_{g \in G} \alpha(g) \cdot \overline{\chi_\rho(g)} \\
    & = |G| \cdot \langle \alpha, \chi_\rho \rangle = 0 \label{equ:numirr.numconj.last}
\end{align}
\end{subequations}
Therefore, it must be $\phi_{\alpha, \rho} = 0$ for any irreducible representation.
The equality in \Cref{equ:numirr.numconj.last} is due to our assumption that $\langle \alpha, \rho \rangle = 0$ for all irreducible representations $\rho$.
Since all representations are composed of irreducible representations via the direct sum, $\phi_{\alpha, \rho} = 0$ for all representations.
So this is also true for the regular representation.
The regular representation maps all elements to linearly independent permutation matrices.
For $\phi_{\alpha, \rho} = \sum_{g \in G} \alpha(g) \cdot \rho(g^{-1}) = 0$ to be true for the regular representation, $\alpha$ must be $\alpha = 0$~\cite{fulton2013}.

\textbf{Character tables} are just a way to write down characters of representations of groups.
All representations in one table are representations of the same group.
Typically, all \textit{simple} representations of the group are listed in the table.
Each row corresponds to one representation.
Each column corresponds to one conjugacy class of the group.
The classes are written on top of every column and above that, the number of group elements of that conjugacy class is written.
The number of conjugacy classes and the number of simple representations is always the same, so the portion of the table showing the characters is always square.
In the top left corner, the name of the group is written~\cite{fulton2013}.
