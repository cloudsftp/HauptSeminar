\subsection{Character Tables}

Character tables are the way, to write down representations, which was promised at the beginning of the section.
But first, we have to define what \textbf{characters} are.
Suppose $\rho: G \to \GL(V)$ is a representation of a finite group $G$.
The character of $\rho$ is defined as the map
\begin{subequations}
    \begin{align}
        \chi_\rho: \qquad & G \to \C, \\
        & a \mapsto \Tr(\rho(a))
    \end{align}
\end{subequations}
A representation can be recovered completely from its character.
So now we know, we only have to write down one number per element $g \in G$.
But since the trace of a matrix stays the same after a base change, it is also true that
\begin{align}
    \forall g, h \in G: \qquad & \chi_\rho(h \cdot g \cdot h^{-1}) = \chi_\rho(g)
\end{align}
This tells us that the character $\chi_\rho$ is constant on the conjugacy classes of $G$.
So we only have to write down one number per conjugacy class of $G$.
This saves a lot of space and contains all the information needed to recover the representation~\cite{fulton2013}.

\textbf{Character tables} are just a way to write down characters of representations of groups.
All representations in one table are representations of the same group.
Typically, all \textit{simple} representations of the group are listed in the table.
Each row corresponds to one representation.
Each column corresponds to one conjugacy class of the group.
The classes are written on top of every column and above that, the number of group elements of that conjugacy class is written.
The number of conjugacy classes and the number of simple representations is always the same, so the portion of the table showing the characters is always square.
In the top left corner, the name of the group is written~\cite{fulton2013}.

\begin{table}
    \centering

    \begin{tabular}{r | c  c  c}
                & 1 & 3     & 2     \\
        $S_3$   & 1 & (12)  & (123) \\ \hline
        triv    & 1 & 1     & 1     \\
        sgn     & 1 & -1    & 1     \\
        stand   & 2 & 0     & -1
    \end{tabular}

    \caption{Character table for the group $S_3$}
    \label{tab:reprep.char.table}
\end{table}

\Cref{tab:reprep.char.table} is an example of a character table.
It shows all simple representations of the group $S_3$.
$S_3$ is the group of symmetries of a triangle.
$1$ is the identity element, each representation will map it to the corresponding identity matrix of its vector space.
This means that the character in that column will tell us the dimension of the vector space, the representation maps to, since $\chi_V(1) = \Tr(I_V) = \dim(V)$.
The trivial representation, triv, and the sign representation, sgn, both have dimension one, so their character shows the matrices, which the group elements are mapped onto.
The trivial representation just maps all elements to 1.

\begin{figure}[!h]
    \centering
    
    \begin{tikzpicture}
    \draw   (0, 0) node [anchor=north east] {1}
        --  (60:2) node [anchor=south] {2}
        --  (2, 0) node [anchor=north west] {3}
        --  cycle;
    \end{tikzpicture}

    \caption{Triangle for visualizing $S_3$}
    \label{fig:reprep.char.triangle}
\end{figure}

For the sign representation, we need to know, what the conjugacy classes are and what the notation means.
We think of each element as an action on a triangle as shown in \Cref{fig:reprep.char.triangle}.
The notation will tell us, which corner is mapped where.
The corner with the first number will be mapped to the corner with the second number.
If there is a third number, the corner with the second number will be mapped to the corner with the third number.
And finally, the corner with the last number will be mapped to the corner with the first number.
So the group element (12) will just swap corners 1 and 2 and the group element (123) will rotate the triangle clockwise.
Every element in the conjugacy group of (12) will invert the order of numbers from clockwise to counterclockwise and vice versa, hence the sign representation maps its elements to -1.
The identity and rotations will leave the order intact and therefore the sign representation maps 1 and every element in the conjugacy group of (123) to 1.

Assume, that every corner of the triangle in \Cref{fig:reprep.char.triangle} is associated with a value $x_i$ where $x_1 + x_2 + x_3 = 0$.
This subspace of $\C^3$ can be thought of a 2-dimensional vector space with the basis $\{e_1 - e_2, e_2 - e_3\}$ with
\begin{align*}
    x' = \begin{pmatrix}
        x_1 - x_2 \\
        x_2 - x_3
    \end{pmatrix}
\end{align*}
The elements of $S_3$ map the values of $x_i$ between the corners as described above.
So the standard representation maps 1 to I because all $x_i$s stay the same and therefore also $x'_1 = x_1 - x_2$ and $x_2' = x_2 - x_3$.
I will write $\stand((12))$ as $s^{(12)}$ in the following for readability.
The action (12) swaps the values of $x_1$ and $x_2$.
So $s^{(12)}_{00} = -1$ and $s^{(12)}_{01} = 0$, since $x'_1$ is just negated.
For $x'_2$ it is a little more complicated, the elements of the matrix must satisfy the equality
\begin{align*}
    x_1 - x_3 & = s^{(12)}_{10} \cdot (x_1 - x_2) + s^{(12)}_{11} \cdot (x_2 - x_3) \\
    & = x_1 \cdot s^{(12)}_{10} + x_2 \cdot (s^{(12)}_{11} - s^{(12)}_{10}) + x_3 \cdot s^{(12)}_{11} \\
\end{align*}
So $s^{(12)}_{10} = s^{(12)}_{11} = 1$ and the resulting matrix is
\begin{align*}
    s^{(12)} & = \begin{pmatrix}
        -1 & 0 \\
        1 & 1
    \end{pmatrix}
\end{align*}
Its trace is 0.
We don't need to compute the matrices for the other two elements in the same conjugacy class, since the character is constant on the conjugacy classes as mentioned above.
Now we computed the values of the character of the standard representation of $S_3$ for the conjugacy classes of 1 and (12).
The value for the conjugacy class (123) is missing.
I will write $\stand((123))$ as $s^{(123)}$ in the following for readability.
For to get the correct value in $x'_1$, the elements of the matrix must satisfy the following equality
\begin{align*}
    x_2 - x_3 & = s^{(123)}_{00} \cdot (x_1 - x_2) + s^{(123)}_{01} \cdot (x_2 - x_3)
\end{align*}
So $s^{(123)}_{00} = 0$ and $s^{(123)}_{01} = 1$.
For $x'_2$ it is more complicated and the remaining elements of the matrix have to satisfy the equality
\begin{align*}
    x_3 - x_1 & = s^{(123)}_{10} \cdot (x_1 - x_2) + s^{(123)}_{11} \cdot (x_2 - x_3) \\
    & = x_1 \cdot s^{(123)}_{10} + x_2 \cdot (s^{(123)}_{11} - s^{(123)}_{10}) + x_3 \cdot s^{(123)}_{11}
\end{align*}
So $s^{(123)}_{10} = s^{(123)}_{11} = -1$ and  the resulting matrix with trace -1 is
\begin{align*}
    s^{(123)} & = \begin{pmatrix}
        0 & 1 \\
        -1 & -1
    \end{pmatrix}
\end{align*}
