\subsection{Character Tables}

Character tables are the way, to write down representations, which was promised at the beginning of the section.
But first, we have to define what \textbf{characters} are.
Suppose $\rho: G \to \GL(V)$ is a representation of a finite group $G$.
The character of $\rho$ is defined as a map $\chi_V: G \to \C$ with
\begin{align}
    \forall a \in G: \qquad & \chi_V(a) = \Tr(\rho(a))
\end{align}
A representation can be recovered completely from its character.
So now we know, we only have to write down one number per element $g \in G$.
But since the trace of a matrix stays the same after a base change, it is also true that
\begin{align}
    \forall g, h \in G: \qquad & \chi_V(h \cdot g \cdot h^{-1}) = \chi_V(g)
\end{align}
This tells us that the character $\chi_V$ is constant on the conjugacy classes of $G$.
So we only have to write down one number per conjugacy class of $G$.
This saves a lot of space and contains all the information needed to recover the representation~\cite{fulton2013}.
