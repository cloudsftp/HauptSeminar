\subsection{(Semi-)simple Representations and Shur's Lemma}
\label{sec:reprep.simp}

As mentioned at the beginning of this section, the ``elementary'' representations we want, are representations that cannot be created from smaller representations by combining them.
This property is called \textbf{irreducibility} and is better defined in the following way.
A representation $\rho: G \to \GL(V)$ is \textbf{irreducible} (also called \textbf{simple}) if there is no non-trivial vector subspace $U \subset V$ that is invariant under $\rho$.
$U$ is invariant under $\rho$ if for every $a \in G$ and $u \in U$ the following condition is satisfied.
\begin{align}
    \rho(a) \cdot u \in U
\end{align}
Trivial vector subspaces of $V$ are $V$ itself and the space consisting only of the additive neutral $\{0_V\}$~\cite{hein2013}.

Such a vector subspace $U$ would be called \textbf{$G$-invariant}.
And if it existed, there would be a \textbf{subrepresentation} $\rho': G \to \GL(U)$ of $\rho$.
The specific images of $\rho'$ can be obtained from the actions of the larger representation $\rho$ on the vector subspace $U$.
Both $U$ and $V$ are also called \textbf{$G$-modules}.
The formal definition of a $G$-module is a vector space $V$ over a field $F$ with an operation $\cdot: G \times V \to V$ with the following properties.
\begin{subequations}
    \begin{align}
        \forall a, b \in G,\ x \in V: \qquad & (a \cdot b) \cdot x = a \cdot (b \cdot x) \\
        \forall x \in V: \qquad & e \cdot x = x \\
        \forall a \in G,\ \beta \in F,\ x \in V: \qquad & a \cdot (\beta \cdot x) = \beta \cdot (a \cdot x) \\
        \forall a \in G,\ x, y \in V: \qquad & a \cdot (x + y) = a \cdot x + a \cdot y
    \end{align}
\end{subequations}
where $e$ is the identity element of the group $G$.
Every $G$-module $V$ induces a representation $\rho: G \to \GL(V)$ where $\rho(a) \cdot x = a \cdot x$.
$\rho(a)$ is linear for any action $a \in G$ thanks to the properties of the operation $\cdot: G \times V \to V$ defined above.
If a representations $\rho: G \to \GL(V)$ is irreducible, the $G$-module $V$ is also irreducible~\cite{hein2013}.

If $U$ is a $\rho$-invariant vector subspace of $V$, then there is a complementary vector subspace $U'$ that is also $\rho$-invariant and for which $V = U \oplus U'$.
For the proof of the statement, we now think of $U$ and $V$ as $G$-modules.
We choose some arbitrary vector subspace $W$ that is a complement of $U$.
Let $\pi_0: V \to U$ be the projection map of $V$ onto $U$.
That means $\pi_0(u + w) = u$ for any $u \in U, w \in W$.
If we average that map over $G$, we get
\begin{subequations}
\begin{align}
    \pi: \qquad & V \to U, \\
    & v \mapsto \sum_{g \in G} g \cdot \pi_0(g^{-1} \cdot v)
\end{align}
\end{subequations}
Where $g^{-1}$ acts on the $G$-module $V$ and $g$ acts on the $G$-module $U$ after the projection with $\pi_0$.
The kernel of this map $\ker \pi$ is also $G$-invariant because for any $h \in G, w' \in \ker \pi$
\begin{subequations}
\begin{align}
    \pi(h \cdot w') & = h \cdot h^{-1} \sum_{g \in G} g \cdot \pi_0(g^{-1} \cdot (h \cdot w')) \\
    & = h \cdot \sum_{g \in G} (h^{-1} \cdot g) \cdot \pi_0((g^{-1} \cdot h) \cdot w') \label{equ:proof.maschke.hg} \\
    & = h \cdot \sum_{g' \in G} g' \cdot \pi_0(g'^{-1} \cdot w') \label{equ:proof.maschke.gprime} \\
    & = h \cdot \pi(w') = h \cdot 0 = 0
\end{align}
\end{subequations}
Where the step from \Cref{equ:proof.maschke.hg} to \Cref{equ:proof.maschke.gprime} is due to $g' = h^{-1} \cdot g$, whichs inverse is the inverse of each element $h^{-1}$ and $g$ applied in reversed order, so $g'^{-1} = g^{-1} \cdot h$.
This kernel $\ker \pi$ is complementary to $U$ because per construction of $\pi$, $\pi(u) = u$ for any $u \in U$ and $\pi(\pi(v)) = \pi(v)$ for any $v \in V$.
So we can set $U' = \ker \pi$ and then $V = U \oplus U'$ with both summands being invariant under $\rho$~\cite{fulton2013}.

From this observation, we can prove, what is referred to as \textbf{Maschke's Theorem}.
If we have a reducible representation, we can successively split it into smaller representations until we only have irreducible representations left.
Therefore, any representation there exists a \textbf{decomposition}
\begin{align}
    \rho & = \rho_1^{\oplus a_1} \oplus \rho_2^{\oplus a_2} \oplus \ldots \oplus \rho_k^{\oplus a_k} 
\end{align}
where $\rho_i$ are distinct irreducible representations of $G$.
This property of representations of finite groups was hinted at earlier in \Cref{sec:basics.rep.faith} and is called \textbf{semisimplicity}.
It is only true for vector spaces over a field, that is closed algebraically~\cite{fulton2013}.

Another important theorem in representation theory is \textbf{Shur's Lemma}.
It makes statements about irreducible representations.
If there are two irreducible representations $\rho: G \to \GL(V)$ and $\rho': G \to \GL(V')$ of a group $G$ and $f: V \to V'$ is an intertwiner (see \Cref{sec:basics.equ} for the definition) of these representations, then the following statements are true.
\begin{enumerate}
    \item Either $f$ is an isomorphism (and the representations are equivalent), or $f = 0_{V'}$
    \item If $V = V'$, then $\exists \lambda \in \C:\ f = \lambda \cdot I$
\end{enumerate}
For the proof of the first statement, we think about the kernel and image of the intertwiner $f$.
Since $f$ is an intertwiner, both the kernel $\ker f$ and the image $\Im f$ have to be $G$-invariant.
If the image $\Im f \neq \left\{ 0_{V'} \right\}$, then it must be $\Im f = V'$, since $\rho'$ is irreducible.
This means that $f$ is surjective.
The kernel must be $\ker f = \left\{ 0_V \right\}$ since $\rho$ is irreducible and it can't be $\ker f = V$, since the image is $\Im f = V'$.
Therefore, $f$ is also injective.
Combining this with the fact that $f$ is surjective, makes $f$ bijective, so it is invertible and therefore an isomorphism.

For the second statement, we take advantage of the fact, that $\C$ is algebraically closed and $f$ is a linear map of a finite vector space onto itself.
$f$ therefore has an eigenvector $v \neq 0_V$ with a corresponding eigenvalue $\lambda$.
$h = f - \lambda \cdot I$ is $G$-linear since it is the sum of $G$-linear maps and $h(v) = 0$.
This means that $h$ has a non-zero kernel and so by the first statement, $h$ must be $h = 0_{V'}$.
From this, we can conclude, that $f = \lambda \cdot I$.

With this theorem, we can prove that the decomposition mentioned above is \textbf{unique}.
Remember, our decomposition is $\rho = \bigoplus_{i=1}^k \rho^{\oplus a_i}_i$.
Let $\rho' = \bigoplus_{j=1}^k \rho'^{\oplus b_j}_j$ be another decomposition and $f: V \to W$ an intertwiner between $\rho: G \to \GL(V)$ and $\rho': G \to \GL(W)$.
Since $f$ is $G$-linear, it must map every summand $V_i^{\oplus a_i}$ of $V$ to a summand $W_j^{\oplus b_j}$, for which $V_i^{\oplus a_i} \equiv W_j^{\oplus b_j}$.
Otherwise $f$ must be $f = 0_W$ by Shur's Lemma.
Therefore, the decomposition is unique up to the reordering and equivalence of summands.
