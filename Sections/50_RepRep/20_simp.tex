\subsection{Simple Representations}

To define simple representations, we first need to understand what irreducible representations are.
A representation $\rho: G \to \GL(V)$ is \textbf{irreducible} (also called \textbf{simple}), if there is no non-trivial vector subspace $U \subset V$ exists with $\rho(a) \cdot U \subset U$ for all actions $a \in G$.
Trivial vector subspaces of $V$ are $V$ itself and the space consisting only of the additive neutral $\{0_V\}$~\cite{hein2013}.

Such a vector subspace $U$ would be called \textbf{$G$-invariant}.
Both $U$ and $V$ are also called \textbf{$G$-modules}.
The formal definition of a $G$-module is a vector space $V$ over a field $F$ with an operation $\cdot: G \times V \to V$ with the following properties.
\begin{subequations}
    \begin{align}
        \forall a, b \in G,\ x \in V: \qquad & (a \cdot b) \cdot x = a \cdot (b \cdot x) \\
        \forall x \in V: \qquad & e \cdot x = x \\
        \forall a \in G,\ \beta \in F,\ x \in V: \qquad & a \cdot (\beta \cdot x) = \beta \cdot (a \cdot x) \\
        \forall a \in G,\ x, y \in V: \qquad & a \cdot (x + y) = a \cdot x + a \cdot y
    \end{align}
\end{subequations}
where $e$ is the identity element of the group $G$.
Every $G$-module $V$ induces a representation $\rho: G \to \GL(V)$ where $\rho(a) \cdot x = a \cdot x$.
$\rho(a)$ is linear for any action $a \in G$ thanks to the properties of the operation $\cdot: G \times V \to V$ defined above~\cite{hein2013}.

\todo[inline]{later}
And if it existed, $\rho': G \to \GL(U)$ would be a \textbf{subrepresentation} of $\rho$.
