\subsection{Equivalence of Representations}
\label{sec:basics.equ}

In the previous section, we saw how the regular representation can be transformed to consist of block matrices.
This was achieved through a change of basis.
Informally we can say, that they are equivalent because, through the change of basis, no structure is lost.
This poses the question, of whether there is a general definition when two representations are equivalent.

First, we define the concept of \textbf{intertwiners}, also called \textbf{equivariant maps} between representations.
These are just homomorphisms between representations, meaning they preserve the structure of the representations.
A map $f: V \to V'$ is an intertwiner of two representations $\rho: G \to \GL(V)$ and $\rho': G \to \GL(V')$ of a group $G$, iff the diagram in \Cref{fig:main.what.equ-cd} commutes~\cite{fuchs2003}.

\begin{figure}[h]
    \centering
    \begin{tikzcd}
        V
            \arrow[r, "f"]
            \arrow[d, "\rho(a)"]
        & V'
            \arrow[d, "\rho'(a)"]
        \\
        V   \arrow[r, "f"]
        & V'
    \end{tikzcd}
    \caption{Commuting diagram for intertwiners}
    \label{fig:main.what.equ-cd}
\end{figure}

The two representations $\rho$ and $\rho'$ are \textbf{equivalent}, iff there is an intertwiner $f$ that is invertible.
If the intertwiner is invertible, it is an isomorphism and the representations are isomorphic to each other~\cite{hein2013,fuchs2003}.
This is exactly what a base change is, so in our example from above, $f$ would be the map represented by the base change matrix $A$.
