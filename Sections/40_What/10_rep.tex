\subsection{Representations}

The \textbf{representation} $\rho$ of a group $G$ is defined as a group homomorphism $\rho: G \to \GL(V)$.
This means that $\rho$ preserves the structure of $G$.
So the diagram in \Cref{fig:main.what.rep-cd} commutes.
As you can infer from the diagram, the representation $\rho$ does not have to preserve \textit{all} structure of $G$.
The simplest representation therefore only has to preserve the structure, that every $g \in G$ is an element of $G$.
This is ridiculously trivial and you can see that it is a valid representation for any group $G$.
It is called the trivial representation and is defined as $\triv: G \to \GL(V): g \mapsto \id_V$.
\begin{figure}[h]
    \centering
    \begin{tikzcd}
        G \times G
            \arrow[r, "\rho \times \rho"]
            \arrow[d, "\cdot"]
        & \GL(V) \times \GL(V)
            \arrow[d, "\cdot"]
        \\
        G   \arrow[r, "\rho"]
        & \GL(V)
    \end{tikzcd}
    \caption{Commuting diagram for representations}
    \label{fig:main.what.rep-cd}
\end{figure}
