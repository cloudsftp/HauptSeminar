\subsection{Representations}

The \textbf{representation} $\rho$ of a group $G$ is defined as a group homomorphism $\rho: G \to \GL(V)$.
This means that $\rho$ preserves the structure of $G$.
So the diagram in \Cref{fig:main.what.rep-cd} commutes.
As you can infer from the diagram, the representation $\rho$ does not have to preserve \textit{all} structure of $G$.
The simplest representation therefore only has to preserve the structure, that every $g \in G$ is an element of $G$.
This is ridiculously trivial and you can see that it is a valid representation for any group $G$.
It is called the \textbf{trivial representation} and is defined as $\triv: G \to \GL(V): g \mapsto \id_V$.

A representation that preserves \textit{all} structure of the group $G$ it represents, is called \textbf{faithful}.
Such a representation $\rho$ has to be injective, meaning that $\rho(a) = \rho(b) \implies a = b$ for any $a, b \in G$.
Since $\rho$ is injective, we can define an inverse map on its image $\rho^{-1}: \Ima(\rho) \to G: \rho^{-1}(\rho(g)) = g$.
For this inverse map, the diagram in \Cref{fig:main.what.faith-rep-cd} also commutes.

\begin{figure}[h]
    \begin{subfigure}{.5 \textwidth}
        \centering
        \begin{tikzcd}
            G \times G
                \arrow[r, "\rho \times \rho"]
                \arrow[d, "\cdot"]
            & \GL(V) \times \GL(V)
                \arrow[d, "\cdot"]
            \\
            G   \arrow[r, "\rho"]
            & \GL(V)
        \end{tikzcd}
        \caption{General case}
        \label{fig:main.what.rep-cd}
    \end{subfigure}
    \begin{subfigure}{.5 \textwidth}
        \centering
        \begin{tikzcd}
            \Ima(\rho) \times \Ima(\rho)
                \arrow[r, "\rho^{-1} \times \rho^{-1}"]
                \arrow[d, "\cdot"]
            & G \times G
                \arrow[d, "\cdot"]
            \\
            \Ima(\rho) \arrow[r, "\rho^{-1}"]
            & G
        \end{tikzcd}
        \caption{Faithful representations}
        \label{fig:main.what.faith-rep-cd}
    \end{subfigure}
    \caption{Commuting diagrams for representations}
\end{figure}
