\subsection{Infinite Representation}

An example of an infinite-dimensional representation of a group would be the infinite-dimensional representation of the group $\text{SL}_2(\mathbb{R})$.
It is the group of all $2 \times 2$ real matrices with a determinant of one.
From this definition, you can immediately see, that there exists at least one finite-dimensional representation of this group.

To describe this infinite-dimensional representation, we first need to find generators of the group $\text{SL}_2(\mathbb{R})$.
In this case, there are three and we will call them $E$, $F$, and $H$.
They need to satisfy the following equations:
\begin{subequations}
    \begin{align}
        [H, E] & = 2E \\
        [H, F] & = -2F \\
        [E, F] & = H
    \end{align}
\end{subequations}
where $[A, B] = A \cdot B - B \cdot A$ is the so called commutator of $A$ and $B$.
The matrices satisfying these equations are
\begin{align}
    H = -i \cdot \begin{pmatrix}
        0 & 1 \\
        -1 & 0
    \end{pmatrix}, \qquad
    E = \frac{1}{2} \cdot \begin{pmatrix}
        1 & i \\
        -i & 1
    \end{pmatrix}, \text{ and} \qquad
    F = \frac{1}{2} \cdot \begin{pmatrix}
        1 & -i \\
        -i & 1
    \end{pmatrix} 
\end{align}

For the representation, we need an infinite-dimensional vector space $V$.
We don't need to think about what it looks like, just assume it has an infinite number of basis vectors $\{\ldots, v_{-1}, v_0, v_1, \ldots\}$.
We now construct a representation on a subspace of this vector space $V$ by choosing a $\lambda \in \C$.
The representation $\rho(a) \in \GL(V)$ of an action $a \in \text{SL}_2(\mathbb{R})$ is then given by
\begin{subequations}
    \begin{align}
        \rho(H) \cdot v_n & = n \cdot v_n \\
        \rho(E) \cdot v_n & = \frac{\lambda + n + 1}{2} \cdot v_{n + 2} \\
        \rho(F) \cdot v_n & = \frac{\lambda - n - 1}{2} \cdot v_{n - 2}
    \end{align}
\end{subequations}
where $n \in \mathbb{Z}$.
To obtain the representation $\rho(a)$ for any action $a \in \text{SL}_2(\mathbb{R})$, you first need to write it as some combination of $H$, $E$, and $F$ then you can combine the given linear for every element given above like you combined the elements.
From these three formulas, one can see that the resulting maps on $V$ are indeed linear.
Also, we don't need all basis vectors, we only need either all the ones with even indices or the ones with odd indices.

These representations are irreducible, iff neither $\rho(E) \cdot v_n = 0$ nor $\rho(F) \cdot v_n = 0$ for some $n$.
Otherwise, there would be an opportunity to cut the vector space around that index $n$.
Such a thing happens if $\lambda \in \mathbb{Z}$.
So if we choose $\lambda \in \C \setminus \mathbb{Z}$, the representation above will be infinite and irreducible~\cite{borcherds2011}.
