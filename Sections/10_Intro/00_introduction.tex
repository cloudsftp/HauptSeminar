\section{Introduction}

Representation theory is a mathematical branch, that investigates the structure of groups.
It does so by \textit{representing} group elements as linear maps in vector spaces.
Since group elements are thought of as actions acting on something, it makes sense to represent them by maps, and not elements of vector spaces.
The idea is to represent group elements as their actions on some vector space.
The vector space is then also called a $G$-module, and the mapping of the group elements to the linear maps is called the representation.

Linear algebra is a well-studied branch of mathematics.
And there are a lot of algorithms for tackling problems in this field numerically.
Representing groups as linear maps, therefore, makes it easier to study them.
For example, Burnside's theorem was proven using representation theory in 1904~\cite{burnside1904groups}.
It took roughly 70 years to find a proof without using representation theory.

\todo[inline]{quantum mechanics}

There are representations of groups using infinite-dimensional vector spaces, as well as finite-dimensional vector spaces.
In this document, we will focus on the case of finite-dimensional vector spaces since this is easier to understand and most readers are already familiar with finite-dimensional vector spaces.
The vector spaces used in this document are of the form $\C^n$ for reasons discussed later.
