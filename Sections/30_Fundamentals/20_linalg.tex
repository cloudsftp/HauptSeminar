\subsection{Linear Algebra}

A \textbf{vector space} over a field $F$ is a set $V$ with two operations, addition $+: V \times V \to V$ and scalar multiplication $\cdot: F \times V \to V$, that satisfy the following axioms.
\begin{subequations}
\begin{align}
    \forall u, v, w \in V: \qquad & u + (v + w) = (u + v) + w & \text{associativity of vector addition} \\
    \forall u, v \in V: \qquad & u + v = v + u & \text{commutativity of vector addition} \\
    \exists 0_V \in V\ \forall v \in V: \qquad & v + 0_V = v & \text{identity element of vector addition} \\
    \forall v \in V\ \exists -v \in V: \qquad & -v + v = 0_V & \text{inverse element of vector addition} \\
    \forall v \in V\ a, b \in F: \qquad & a \cdot (b \cdot v) = (a \cdot b) \cdot v & \text{compatibility of multiplications\footnotemark} \\
    \forall v \in V: \qquad & 1 \cdot v = v & \text{identity element of scalar multiplication} \\
    \forall u, v \in V\ a \in F: \qquad & a \cdot (u + v) = a \cdot u + a \cdot v & \text{distributivity of scalar multiplication\footnotemark} \\
    \forall v \in V\ a, b \in F: \qquad & (a + b) \cdot v = a \cdot v + b \cdot v & \text{distributivity of scalar multiplication\footnotemark}
\end{align}
\end{subequations}
\footnotetext{compatibility of scalar multiplication and field multiplication}
\footnotetext{with respect to vector addition}
\footnotetext{with respect to field addition}

A \textbf{linear map} $m$ between vector spaces $V$ and $W$ is a map $m: V \to W$ that satisfies the following two conditions:
\begin{subequations}
\begin{align}
    \forall u, v \in V: \qquad & m(u + v) = m(u) + m(v) & \text{additivity} \\
    \forall c \in \C, v \in V: \qquad & m(c \cdot v) = c \cdot m(v) & \text{homogeneity}
\end{align}
\end {subequations}
In our case both vector spaces $V$ and $W$ always have the same, so $V = W$.
The subset of all \textbf{invertible} linear maps between $V$ and itself is denoted as $\GL(V)$.
\\

In most cases, the representations of groups are over finite-dimensional vector spaces.
If they are finite-dimensional, they will be vector spaces over the field of complex numbers $\C$ in this work.
A $n$-dimensional complex vector space is denoted as $\C^n$.

If a linear map is between two finite-dimensional vector spaces and both of those have a defined basis, it can be represented by a \textbf{matrix}.
In our finite-dimensional cases, both vector spaces are $\C^n$ and we define the standard basis $\{e_i | 0 \leq i < n\}$ where $e_i = \begin{pmatrix} 0 & \ldots & 0 & 1 & 0 & \ldots & 0 \end{pmatrix}^T$ is a vector with all zeros except the $i$-th element.
So we can represent the linear maps as matrices in the finite-dimensional cases in this paper.
\textbf{Matrix multiplication} $\cdot: \C^{a \times b} \times \C^{b \times c} \to \C^{a \times c}$ is equivalent to the composition of the linear maps, they represent.
It is defined as
\begin{align}
    A \cdot B = \left(\sum_k a_{ik} \cdot b_{kj}\right)_{ij}
\end{align}
A matrix $A \in \C^{n \times n}$ is \textbf{invertible}, if and only if it represents an invertible linear map.
This is the case if and only if there exists a matrix $A^{-1} \in C^{n \times n}$, such that $A^{-1} \cdot A = I$, where $I$ is the identity matrix. 
The identity matrix is defined as $I = (\delta_{kj})_{kj}$ where $\delta_{kj}$ is the Kronecker delta defined as
\begin{align}
    \delta_{kj} = \begin{cases}
        1 & \text{, if } k = j \\
        0 & \text{, else}
    \end{cases}
\end{align}

A linear map from a complex vector space to itself is represented as a \textbf{square} matrix.
The following two concepts only exist for this kind of map.
The \textbf{trace} of a square matrix $\Tr: \C^{n \times n} \to \C$ is defined as
\begin{align}
    \Tr(A) = \sum_i a_{ii}
\end{align}

For a linear map $A$ from a complex vector space $\C^n$ to itself, there is at least one vector $v \in \C^n$ and a scalar $\lambda \in \C$ so that
\begin{align}
    A \cdot v & = \lambda \cdot v
\end{align}
$v$ is called \textbf{Eigenvector} of $A$ and $\lambda$ is called \textbf{Eigenvalue} of $A$.
The set of all Eigenvalues of a matrix is denoted $\sigma_A = \left\{ \lambda_i \right\}$.
If $v_i \in \C^n$ is an Eigenvector with the Eigenvalue $\lambda_i$, then every multiple $\kappa \cdot v_i$ with $\kappa \in \C$ is also Eigenvector of $A$ with the Eigenvalue $\lambda_i$.
There may be other vectors that also satisfy the condition for the same $\lambda_i$.
All the vectors, that do form a sub vector space denoted
\begin{align}
    E_{\lambda_i} & = \left\{ v \in \C^n | A \cdot v = \lambda_i \cdot v \right\}
\end{align}

%% start needed?
\todo[inline]{Is scalar product nescessary?}

On $\C^n$, the \textbf{scalar product} (also called dot product) $\langle \cdot, \cdot \rangle: \C^n \times \C^n \to \C$ is defined as
\begin{align}
    \langle v, w \rangle = \sum_i v_i \cdot w_i
\end{align}
It can also be thought of as a matrix multiplication of a row vector and a column vector $\langle v, w \rangle = v^T w$.
%% end needed?

The \textbf{direct sum} of two matrices $\oplus: \C^{a \times b} \times \C^{c \times d} \to \C^{(a + c) \times (b + d)}$ is defined as
\begin{align}
    A \oplus B = \begin{pmatrix}
        A & 0 \\
        0 & B
    \end{pmatrix}
\end{align}
where each zero stands for a matrix that only has zeros.

\todo[inline]{block matrix}


%% start needed?
\todo[inline]{Is tensor product and symmetric powers necessary?}
The \textbf{tensor product} of two matrices $\otimes: \C^{a \times b} \times \C^{c \times d} \to \C^{(a \cdot c) \times (b \cdot d)}$ is defined as
\begin{align}
    A \otimes B = \begin{pmatrix}
        a_{1, 1} \cdot B & a_{1, 2} \cdot B & \ldots & a_{1, b} \cdot B \\
        \vdots & \vdots & \ddots & \vdots \\
        a_{a, 1} \cdot B & a_{a, 2} \cdot B & \ldots & a_{a, b} \cdot B
    \end{pmatrix}
\end{align}

\todo[inline]{outer and symmetric powers}

\todo[inline]{Is change of basis necessary?}

A \textbf{change of basis} from one basis $\{v_i\}_i$ to another one $\{w_j\}_j$ in a vector space $C^n$ can be represented by an invertible matrix.
We find this matrix by writing each vector of the new basis $w_j$ as a linear combination of the vectors in the old basis $v_i$.
\begin{align}
    w_j = \sum_i a_{ij} v_i
\end{align}
This gives us the elements of $A = (a_{ij})_{ij}$.
A vector $x \in \C^n$ is transformed from the old to the new basis via a matrix multiplication $x' = A^{-1} \cdot x$.
A linear map between $\C^n$ and itself represented as a matrix $M \in \C^{n \times n}$ is transformed via two matrix multiplications $M' = A^{-1} \cdot M \cdot A$.
\\

\todo[inline]{still necessary? Do I introduce dual representation later?}

A \textbf{dual vector space} $V^*$ of a vector space $V$ over a field $F$ is the set of all linear maps $m: V \to F$.
In the case of $\C^n = \C^{n \times 1}$ we can think of it as row vectors $(\C^n)^* = \C^{1 \times n}$.
Multiplying a row vector with a column vector will result in a number, this follows from the definition of matrix multiplication above.
Since every element in the row vector corresponds to the weight of one element in the column vector, before summing everything up, every linear map $m: \C^n \to \C$ can be represented by exactly one row vector.
