\subsection{Linear Algebra}

A \textbf{vector space} over a field $F$ is a set $V$ with two operations, addition $+: V \times V \to V$ and scalar multiplication $\cdot: F \times V \to V$, that satisfy the following axioms.
\begin{align}
    \forall u, v, w \in V: \qquad & u + (v + w) = (u + v) + w & \text{associativity of vector addition} \\
    \forall u, v \in V: \qquad & u + v = v + u & \text{commutativity of vector addition} \\
    \exists 0_V \in V\ \forall v \in V: \qquad & v + 0_V = v & \text{identity element of vector addition} \\
    \forall v \in V\ \exists -v \in V: \qquad & -v + v = 0_V & \text{inverse element of vector addition} \\
    \forall v \in V\ a, b \in F: \qquad & a \cdot (b \cdot v) = (a \cdot b) \cdot v & \text{compatibility of multiplications} \\
    \forall v \in V: \qquad & 1 \cdot v = v & \text{identity element of scalar multiplication} \\
    \forall u, v \in V\ a \in F: \qquad & a \cdot (u + v) = a \cdot u + a \cdot v & \text{distributivity of scalar multiplication} \\
    \forall v \in V\ a, b \in F: \qquad & (a + b) \cdot v = a \cdot v + b \cdot v &  \text{distributivity of scalar multiplication}
\end{align}

A \textbf{linear map} $m$ between vector spaces $V$ and $W$ is a map $m: V \to W$ that satisfies the following two conditions.
\begin{align}
    \forall u, v \in V: \qquad & m(u + v) = m(u) + m(v) & \text{additivity} \\
    \forall c \in \C, v \in V: \qquad & m(c \cdot v) = c \cdot m(v) & \text{homogeneity}
\end{align}
In our case both vector spaces $V$ and $W$ always have the same, so $V = W$.
The set of all linear maps between $V$ and itself is denoted as $\GL(V)$.
\\

In most cases, the representations of groups are over finite-dimensional vector spaces.
If they are finite-dimensional, they will be vector spaces over the field of complex numbers $\C$ in this work.
A $n$-dimensional complex vector space is denoted as $\C^n$.
Its elements are written as vectors.

\todo[inline]{Matrices}
???
Such a map $M$ can be expressed via a \textbf{matrix}.
This is a core concept in representation theory since the elements of groups will be represented by matrices.
