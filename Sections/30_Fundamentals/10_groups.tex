\subsection{Group Theory}

A \textbf{group} $(G, \cdot)$ is a set $G$ with an operation $\cdot: G \times G \to G$ that satisfies the following axioms.
\begin{align}
    \forall a, b, c \in G: \qquad & a \cdot (b \cdot c) = (a \cdot b) \cdot c & \text{associativity} \\
    \exists e \in G \forall g \in G: \qquad & e \cdot g = g \cdot e = g & \text{neutral element} \\
    \forall g \in G \exists g^{-1} \in G: \qquad & g^{-1} \cdot g = g \cdot g^{-1} = e & \text{inverse element}
\end{align}
If a group also satisfies commutativity, it is called an abelian group.
\begin{align}
    \forall a, b \in G: \qquad & a \cdot b = b \cdot a & \text{commutativity}
\end{align}

\todo[inline]{Examples for groups?}

A \textbf{topological group} $(G, \cdot)$ is a group where both the operation $\cdot: G \times G \to G: (x, y) \mapsto xy$ and the inverse map $^{-1}: G \to G: x \mapsto x^{-1}$ are continuous.
This is the case, iff the map $m: G \times G \to G: (x, y) \mapsto xy^{-1}$ is continuous.
$m$ is continuous, iff for all $x, y \in G$ and any neighbourhood $W$ of $xy^{-1}$ there are neighbourhoods $U$ of $x$ and $V$ of $y$, such that $U \cdot V^{-1} \subseteq W$.

\todo[inline]{Examples for topological groups?}

A \textbf{group homomorphism} is a map $\phi: G \to H$ between two groups $(G, \cdot)$ and $(H, *)$, such that
\begin{align}
    \phi(a \cdot b) = \phi(a) * \phi(b)
\end{align}
Or the diagram in \Cref{fig:fundamentals.groups.hom-cd} commutes.
Such diagrams are common in category theory and representation theory.
Understanding them will help in later sections.
\begin{figure}[h]
    \centering
    \begin{tikzcd}
        G \times G
            \arrow[r, "\phi \times \phi"]
            \arrow[d, "\cdot"]
        & H \times H
            \arrow[d, "*"]
        \\
        G   \arrow[r, "\phi"]
        & H
    \end{tikzcd}
    \caption{Commuting diagram for group homomorphisms}
    \label{fig:fundamentals.groups.hom-cd}
\end{figure}

A \textbf{field} $(F, +, -, 0, 1)$ is a set $F$ with two operations, addition $+: F \times F \to F$ and multiplication $\cdot: F \times F \to F$, such that $(F, +, 0)$ is an abelian group, $(F \setminus \{0\}, \cdot, 1)$ is a group, and the following axioms holds.
\begin{align}
    \forall a, b, c \in F: \qquad & a \cdot (b + c) = a \cdot b + a \cdot c & \text{left distributive property} \\
    \forall a, b, c \in F: \qquad & (a + b) \cdot c = a \cdot c + b \cdot c & \text{right distributive property}
\end{align}
For this work, the only necessary field is the field of the complex numbers $(\C, +, \cdot, 0, 1)$.
When referring to this field in the following, only $\C$ is written.
