\sectionframe{Basics of Representation Theory}

\section{Basics}

\begin{frame}{Definition of a Representation}
    \large
    A \textbf{representation} of a group $G$ is a map
    \Large
    \begin{align*}
        \rho: \quad G \to \GL(V)
    \end{align*}
    
    \normalsize
    \pause
    that is a \textbf{homomorphism} (\textbf{preserves structure})

    \pause
    i.e. the following diagram \textbf{commutes}
    \begin{figure}
        \centering
        \begin{tikzcd}[ampersand replacement=\&]
            G \times G
                \arrow[r, "\rho \times \rho"]
                \arrow[d, "\cdot"]
            \& \GL(V) \times \GL(V)
                \arrow[d, "\cdot"]
            \\
            G   \arrow[r, "\rho"]
            \& \GL(V)
        \end{tikzcd}
    \end{figure}
    {\hspace*{\fill} \cite{hein2013}}

\end{frame}

\begin{frame}{The Trivial Representation}
    \large
    Maps all elements to the identity map
    \Large
        \pause
    \begin{align*}
        \triv: \quad & G \to \GL(\C^1) = \C \setminus \{0\} \pause , \\
        & a \mapsto I = 1
    \end{align*}
    
    \normalsize
    {\hspace*{\fill} \cite{hein2013}}
\end{frame}

\begin{frame}{Faithful Representations}
    \large
    Every group element $a \in G$ has a \textbf{different image} of $\rho: G \to \GL(V)$

    \pause
    So $\rho$ is \textbf{injective}
    \Large
    \begin{align*}
        \forall a, b \in G: \quad & \rho(a) = \rho(b) \implies a = b
    \end{align*}
    \normalsize
    {\hspace*{\fill} \cite{hein2013}}
\end{frame}

\begin{frame}{Equivalence of Representations}
    \large
    $f: V \to V'$ is an \textbf{intertwiner} of $\rho: G \to \GL(V)$ and $\rho': G \to \GL(V')$,
    
    if the following diagram \textbf{commutes} for each group element $a \in G$
    
    \begin{figure}[h]
        \centering
        \begin{tikzcd}[ampersand replacement=\&]
            V
                \arrow[r, "f"]
                \arrow[d, "\rho(a)"]
            \& V'
                \arrow[d, "\rho'(a)"]
            \\
            V   \arrow[r, "f"]
            \& V'
        \end{tikzcd}
    \end{figure}
    
    \vspace*{1em}
    \pause
    If $f$ is \textbf{invertible}, it is an \textbf{isomorphism} and $\rho$ and $\rho'$ are \textbf{equivalent}
    \begin{align*}
        \rho \cong \rho'
    \end{align*}
    
    \normalsize
    {\hspace*{\fill} \cite{fuchs2003}}
\end{frame}
