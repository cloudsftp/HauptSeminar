\sectionframe{Basics of Representation Theory}

\section{Basics}

\begin{frame}{Definition of a Representation}
    \large
    A \textbf{representation} of a group $G$ is a map

    \huge
    \visible<2->{
        \begin{align*}
            \rho: \quad G \to \GL(V)
        \end{align*}
    }
    
    \normalsize
    \visible<3->{
        that is a homomorphism (\textbf{preserves structure})
    }
    
    \visible<4->{
        \begin{figure}
            \centering
            \begin{tikzcd}[ampersand replacement=\&]
                G \times G
                    \arrow[r, "\rho \times \rho"]
                    \arrow[d, "\cdot"]
                \& \GL(V) \times \GL(V)
                    \arrow[d, "\cdot"]
                \\
                G   \arrow[r, "\rho"]
                \& \GL(V)
            \end{tikzcd}
        \end{figure}
    {\hspace*{\fill} \cite{hein2013}}
    }

\end{frame}

\begin{frame}{The trivial Representation}
    \large
    Preserves \textbf{only existence}
    
    \huge
    \begin{align*}
        \visible<2->{
            \triv: \quad & G \to \C^1
        }
        \visible<3->{
            , \\
            & a \mapsto I = 1
        }
    \end{align*}
    
    \normalsize
    \visible<3->{
        {\hspace*{\fill} \cite{hein2013}}
    }
\end{frame}

\begin{frame}{Faithful Representations}
    \large
    Preserve \textbf{all} structure
    
    \vspace*{1em}
    \visible<2->{$\rho$ has to be \textbf{injective}}

    \visible<3->{
        \begin{align*}
            \forall a, b \in G: \quad & \rho(a) = \rho(b) \implies a = b
        \end{align*}
    }

    \normalsize
    \visible<3->{
        {\hspace*{\fill} \cite{hein2013}}
    }
\end{frame}

\begin{frame}{The regular Representation}
    \large
    Easiest to construct \textbf{faithful} Representation

    \vspace*{1em}
    \begin{enumerate}
        \visible<2->{
            \item Choose $\C^{|G|} = \C^n$
        }
        \visible<3->{
            \item Identify every $a \in G$ with one basis vector of $\C^n$
        }
        \visible<4->{
            \item Construct the representation of every $a \in G$ according to its action on $G$
        }
    \end{enumerate}

    \vspace*{1em}
    \visible<5->{
        We let $G$ act on itself

        \normalsize
        {\hspace*{\fill} \cite{fulton2013}}
    }
    
\end{frame}

\begin{frame}{Example of a regular Representation}
    \large
    $\Z/2\Z = \{0, 1\}$ is an \textbf{abelian} group

    \begin{align*}
        \visible<2->{
            \pi: \quad & 0 \mapsto \begin{pmatrix}
                1 \\ 0
            \end{pmatrix}, \quad 1 \mapsto \begin{pmatrix}
                0 \\ 1
            \end{pmatrix} \\
        }
        \visible<3->{
            0: \quad & 0 + 0 = 0,\ 0 + 1 = 1 \\
        }
        \visible<4->{
            \implies & \reg(0) = \begin{pmatrix}
                1 & 0 \\
                0 & 1
            \end{pmatrix} \\
        }
        \visible<5->{
            1: \quad & 1 + 0 = 1,\ 1 + 1 = 0 \\
        }
        \visible<6->{
            \implies & \reg(1) = \begin{pmatrix}
                0 & 1 \\
                1 & 0
            \end{pmatrix}
        }
    \end{align*}
\end{frame}

\begin{frame}{Equivalence of Representations}
    \large
    $f: V \to V'$ is an \textbf{intertwiner} of $\rho: G \to \GL(V)$ and $\rho': G \to \GL(V')$, if
    
    \visible<2->{
        \begin{figure}[h]
            \centering
            \begin{tikzcd}[ampersand replacement=\&]
                V
                    \arrow[r, "f"]
                    \arrow[d, "\rho(a)"]
                \& V'
                    \arrow[d, "\rho'(a)"]
                \\
                V   \arrow[r, "f"]
                \& V'
            \end{tikzcd}
        \end{figure}
        
        commutes
    }
    
    \vspace*{1em}
    \visible<3->{
        If $f$ is invertible, $\rho$ and $\rho'$ are \textbf{equivalent}
        
        \normalsize
        {\hspace*{\fill} \cite{fuchs2003}}
    }
\end{frame}
