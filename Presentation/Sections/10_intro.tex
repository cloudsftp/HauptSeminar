\sectionframe{Introduction}

\section{Introduction}

\begin{frame}{What is Representation Theory?}
    \begin{itemize}
        \visible<1->{\item Representing Group Elements as Linear Maps}
            
        \todo{different motivation}
        \visible<2->{\item First Proof of Burnside's Theorem was using Representations~\cite{burnside1904groups}}
        \visible<3->{\item Applications in Quantum Mechanics and Chemistry}
    \end{itemize}
\end{frame}

\begin{frame}{What is a Group?}
    \Large
    A set $G$ with an operation $\cdot: G \times G \to G$

    \large
    \begin{align*}
        \visible<2->{\forall a, b, c \in G: \qquad & a \cdot (b \cdot c) = (a \cdot b) \cdot c \\}
        \visible<3->{\exists e \in G\ \forall a \in G: \qquad & a \cdot e = e \cdot a = a \\}
        \visible<4->{\forall a \in G\ \exists a^{-1} \in G: \qquad & a^{-1} \cdot a = a \cdot a^{-1} = e}
    \end{align*}
    
    \visible<5->{If also $\forall a, b \in G: a \cdot b = b \cdot a$, then $G$ is called abelian}
\end{frame}

\begin{frame}{Vector Spaces over the Complex Numbers}
    \huge
    \begin{align*}
        \C^n \large
        \visible<2->{
            = \left\{
                \begin{pmatrix}
                    v_1 \\ v_2 \\ \vdots \\ v_n
                \end{pmatrix} | v_i \in \C
            \right\}
        }
    \end{align*}

    \vspace{1em}
    \large
    \visible<3->{
        From now on we call them \textbf{Complex Vector Spaces}
    }
    
\end{frame}

\begin{frame}{What are Linear Maps on Complex Vector Spaces?}
    \Large
    \begin{align*}
        f: \quad & \C^m \to \C^n
    \end{align*}
    \large
    \begin{align*}
        \visible<2->{
            \forall u, v \in \C^m: \quad & f(u + v) = f(u) + f(v) \\
        }
        \visible<3->{
            \forall c \in \C,\ v \in \C^m: \quad & c \cdot f(v) = f(c \cdot v)
        }
    \end{align*}
    
    \visible<4->{
        In this presentation we only need
        \begin{align*}
            f: \quad \C^n \to \C^n
        \end{align*}
    }
    \visible<5->{
        Every such map $f$ can be represented by a \textbf{Complex Matrix} $f \in \C^{n \times n}$
    }
\end{frame}

\begin{frame}{Invertible Linear Maps on Complex Vector Spaces}
    \huge
    \begin{align*}
        \GL(\C^n)
    \end{align*}
    \large
    \visible<2->{
        \begin{align*}
            = \left\{
                f \in \C^{n \times n} | \det(f) \neq 0
            \right\}
        \end{align*}
    }

\end{frame}
