\sectionframe{Introduction}

\section{Introduction}

\begin{frame}{What is Representation Theory?}
    \large
    \begin{itemize}
        \item Representing group elements as linear maps \pause
        \item Finding and classifying elementary representations \pause
        \item Leads to a deeper understanding of groups \pause
        \item First used by Frobenius and Burnside $\sim$1900 \pause
        \item Groups are everywhere in mathematics and physics
    \end{itemize}
\end{frame}

\begin{frame}{Applications of Representation Theory in Cryptography}
    \large
    \begin{itemize}
        \item Understanding groups used in established protocols \pause
        \item Researching new protocols based non-commutative groups and monoids \pause
        \item For example protocols based on $S_n$
    \end{itemize}
    
    \normalsize
    \hspace*{\fill} \cite{khovanov2022monoidal,doliskani2008cryptosystem}
\end{frame}

\begin{frame}{What is a Group?}
    \Large
    A set $G$ with an operation $\cdot: G \times G \to G$ \pause
    \large
    \begin{align*}
        \forall a, b, c \in G: \qquad & a \cdot (b \cdot c) = (a \cdot b) \cdot c \\
        \exists e \in G\ \forall a \in G: \qquad & a \cdot e = e \cdot a = a \\
        \forall a \in G\ \exists a^{-1} \in G: \qquad & a^{-1} \cdot a = a \cdot a^{-1} = e
    \end{align*}

    \pause
    \vspace{1em}
    Conjugacy classes:
    \begin{align*}
        \Cl(a) & = \{g^{-1} \cdot a \cdot g | g \in G \}
    \end{align*}
    
\end{frame}

\begin{frame}{Complex Vector Spaces}
    \huge
    \begin{align*}
        \C^n \large \pause
        = \left\{
            \begin{pmatrix}
                v_1 \\ v_2 \\ \vdots \\ v_n
            \end{pmatrix} | v_i \in \C
        \right\}
    \end{align*}

\end{frame}

\begin{frame}{What are Linear Maps on Complex Vector Spaces?}
    \Large
    \begin{align*}
        f: \quad & \C^m \to \C^n
    \end{align*}
    \large
    \pause
    \begin{align*}
        \forall u, v \in \C^m: \quad & f(u + v) = f(u) + f(v) \\ \pause
        \forall c \in \C,\ v \in \C^m: \quad & c \cdot f(v) = f(c \cdot v)
    \end{align*}
    
    \pause
    In this presentation we only need
    \Large
    \begin{align*}
        f: \quad \C^n \to \C^n
    \end{align*}

    \large
    \pause
    Every such map $f$ can be represented by a \textbf{Complex Matrix} $f \in \C^{n \times n}$
    \vspace{2em}
\end{frame}

\begin{frame}{Traces of Linear Maps}
    The \textbf{Trace} of a matrix $A \in \C^{n \times n}$ is defined as
    \pause
    \begin{align*}
        \Tr(A) & = \sum_{i = 1}^n a_{i, i}
    \end{align*}
    
    \pause
    For all matrices $A, B \in \C^{n \times n}$, the following is true
    \begin{align*}
        \Tr(A \cdot B) & = \Tr(B \cdot A)
    \end{align*}
\end{frame}

\begin{frame}{Eigenspaces of Linear Maps}
    \large
    The \textbf{Eigenvalues} of a matrix $A \in \C^{n \times n}$ are
    \pause
    \begin{align*}
        \sigma(A) =
        \left\{
            \lambda \in \C | \exists v \in \C^n \setminus \{0\}: A \cdot v = \lambda \cdot v
        \right\}
    \end{align*}
    
    \pause
    The vectors satisfying the equation for some $\lambda_i$ compose an \textbf{Eigenspace}
    \pause
    \begin{align*}
        E_{\lambda_i} & = \left\{
            v \in \C^n | A \cdot v = \lambda_i \cdot v
        \right\}
    \end{align*}
    
    \pause
    If all Eigenvalues are \textbf{distinct}, the Eigenspaces are 1-dimensional

    \pause
    So $E_{\lambda_i} = \C \cdot v_i$ for some $v_i \in \C^n$ with $A \cdot v_i = \lambda_i \cdot v_i$ 
\end{frame}

\begin{frame}{Invertible Linear Maps on Complex Vector Spaces}
    \huge
    \begin{align*}
        \GL(\C^n)
    \end{align*}
    \pause
    \large
    \begin{align*}
        = \left\{
            f \in \C^{n \times n} | \det(f) \neq 0
        \right\}
    \end{align*}

\end{frame}
