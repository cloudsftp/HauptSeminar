\sectionframe{Introduction}

\section{Introduction}

\begin{frame}{What is Representation Theory?}
    \begin{itemize}
        \visible<1->{\item Representing Group Elements as Linear Maps}
            
        \todo{different motivation}
        \visible<2->{\item First Proof of Burnside's Theorem was using Representations~\cite{burnside1904groups}}
        \visible<3->{\item Applications in Quantum Mechanics and Chemistry}
    \end{itemize}
\end{frame}

\begin{frame}{What is a Group?}
    \large
    A set $G$ with an operation $\cdot: G \times G \to G$

    \begin{align*}
        \visible<2->{\forall a, b, c \in G: \qquad & a \cdot (b \cdot c) = (a \cdot b) \cdot c \\}
        \visible<3->{\exists e \in G\ \forall a \in G: \qquad & a \cdot e = e \cdot a = a \\}
        \visible<4->{\forall a \in G\ \exists a^{-1} \in G: \qquad & a^{-1} \cdot a = a \cdot a^{-1} = e}
    \end{align*}
    
    \visible<5->{If also $\forall a, b \in G: a \cdot b = b \cdot a$, then $G$ is called abelian}
\end{frame}

\begin{frame}{Vector Spaces over the Complex Numbers}
    \Huge
    \begin{align*}
        \C^n \large
            = \left\{
                \begin{pmatrix}
                    v_1 \\ v_2 \\ \vdots \\ v_n
                \end{pmatrix} | v_i \in \C
            \right\}
    \end{align*}

    \vspace{1em}
    \large
    \visible<3->{
        From now on we call them \textbf{Complex Vector Spaces}
    }
    
\end{frame}

\begin{frame}{Linear Maps on Complex Vector Spaces}
    \large
    \todo{first general, then concrete?}
    \begin{align*}
        m: \quad \C^n \to \C^n
    \end{align*}
    
    \begin{align*}
        \visible<2->{
            \forall u, v \in \C^n: \qquad & m(u + v) = m(u) + m(v) \\
        }
        \visible<3->{
            \forall c \in \C,\ v \in \C^n: \qquad & c \cdot m(v) = m(c \cdot v)
        }
    \end{align*}
    
    \normalsize
    \visible<4->{
        \todo{needed? say just matrices instead?}
        Examples for $\C^2$:
    }
    \begin{align*}
        \visible<5->{
            I = \begin{pmatrix}
                1 & 0 \\
                0 & 1
            \end{pmatrix}, \qquad
        }
        \visible<6->{
            \begin{pmatrix}
                0 & i \\
                -i & 1
            \end{pmatrix}
        }
    \end{align*}
\end{frame}

\begin{frame}{Invertible Linear Maps on Complex Vector Spaces}
    \Huge
    \begin{align*}
        \GL(\C^n)
    \end{align*}
\end{frame}
