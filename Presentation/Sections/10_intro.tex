\sectionframe{Introduction}

\section{Introduction}

\begin{frame}{What is Representation Theory?}
    \begin{itemize}
        \visible<1->{\item Representing Groups as Linear Maps}
        \visible<2->{\item First Proof of Burnside's Theorem was using Representations~\cite{burnside1904groups}}
        \visible<3->{\item Applications in Quantum Mechanics and Chemistry}
    \end{itemize}
\end{frame}

\begin{frame}{What is a Group?}
    A set $G$ with an operation $\cdot: G \times G \to G$

    \begin{align*}
        \visible<2->{\forall a, b, c \in G: \qquad & a \cdot (b \cdot c) = (a \cdot b) \cdot c \\}
        \visible<3->{\exists e \in G\ \forall a \in G: \qquad & a \cdot e = e \cdot a = a \\}
        \visible<4->{\forall a \in G\ \exists a^{-1} \in G: \qquad & a^{-1} \cdot a = a \cdot a^{-1} = e}
    \end{align*}
    
    \visible<5->{If also $\forall a, b \in G: a \cdot b = b \cdot a$, then $G$ is called abelian}
\end{frame}

\begin{frame}{What is a vector space?}
    \Huge
        
    \visible<1->{
        \begin{align*}
            \C^n
        \end{align*}
    }
    
    \normalsize
    \visible<2->{
        Examples for $\C^2$:
    }

    \begin{align*}
        \visible<3->{
            \begin{pmatrix}
                1 \\ 0
            \end{pmatrix}
        }
        \visible<4->{, \qquad
            \begin{pmatrix}
                0 \\ 1 - i
            \end{pmatrix}
        }
    \end{align*}
    
\end{frame}
