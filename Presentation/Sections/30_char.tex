\sectionframe{Classification of Representations}

\section{Classification of Representations}

\begin{frame}{Combining Representations}
    \large
    Direct sum of Vector Spaces over the Complex Numbers:
    \visible<2->{
        \begin{align*}
            \C^n \oplus \C^m & = \C^{n + m}
        \end{align*}
    }
    \visible<3->{Direct sum of Complex Matrices ($A \in \C^{m \times n}, B \in \C^{o \times p}$):}
    \visible<4->{
        \begin{align*}
            A \oplus B & = \begin{pmatrix}
                A & 0 \\
                0 & B
            \end{pmatrix} \in \C^{(m + o) \times (n + p)}
        \end{align*}
    }
    \visible<4->{Direct Sum of Representations ($\rho: G \to \GL(\C^n), \rho': G \to \GL(\C^m)$):}
    \visible<5->{
        \begin{align*}
            (\rho \oplus \rho')(a) & = \rho(a) \oplus \rho'(a)
        \end{align*}
    }

\end{frame}

\begin{frame}{Irreducible Representations}
    \large
    $\rho: G \to \GL(V)$ is \textbf{irreducible},
    
    \visible<2->{
        if there is no non-trivial subspace $U \subset V$ with
    }
    \visible<3->{
        \begin{align*}
            \rho(a) \cdot U & \subset U & \forall a \in G
        \end{align*}
    }
    \visible<4->{
        Also called \textbf{simple}
        
        \normalsize
        {\hspace*{\fill} \cite{hein2013}}
    }

\end{frame}

\begin{frame}{Shur's Lemma}
    \large
    $\rho: G \to \GL(V)$ and $\rho': G \to \GL(V')$ \textbf{irreducible}

    \vspace{1em}
    \visible<2->{
        $f: V \to V'$ \textbf{intertwiner} of $\rho$ and $\rho'$
    }
    
    \vspace{1em}
    \begin{enumerate}
        \visible<3->{
            \item Either $f$ is an \textbf{isomorphism} (so $\rho \cong \rho'$) or $f = 0_{V'}$
        }
        \visible<4->{
            \item If $V = V'$, then $\exists \lambda \in \C:\ f = \lambda \cdot I$
        }
    \end{enumerate}    
    
        \normalsize
    \visible<4->{
        {\hspace*{\fill} \cite{fulton2013}}
    }

\end{frame}

\begin{frame}{Results of Shur's Lemma}
    \large
    \begin{itemize}
        \item Any \textbf{irreducible} representation of an \textbf{abelian} group is 1-dimensional
        \visible<2->{
            \item Any representation $\rho$ of a finite group has a \textbf{unique} decomposition
            \begin{align*}
                \rho & = \rho_1^{\oplus a_1} \oplus \rho_2^{\oplus a_2} \oplus \ldots \oplus \rho_k^{\oplus a_k}
            \end{align*}
        }
    \end{itemize}
    
    \normalsize
    \visible<2->{
        {\hspace*{\fill} \cite{fulton2013}}
    }

\end{frame}

\begin{frame}{Characters}
    \large
    The \textbf{character} of a representation $\rho$ is the map
    \Large
    \begin{align*}
        \visible<2->{
            \chi_\rho: \quad & G \to \C
        }
        \visible<3->{
            , \\
            & a \mapsto \Tr(\rho(a))
        }
    \end{align*}
    
    \large
    \begin{itemize}
        \visible<4->{
            \item $\rho$ completely recoverable from $\chi_\rho$ (up to isomorphism)
        }
        \visible<5->{
            \item \textbf{constant} on the conjugacy classes of $G$
        }
    \end{itemize}

    \vspace{1em}
    \visible<6->{
        Conjugacy classes:
        \begin{align*}
            \Cl(a) & = \{g^{-1} \cdot a \cdot g | g \in G \}
        \end{align*}

        \normalsize
        {\hspace*{\fill} \cite{fulton2013}}
    }
    
\end{frame}

\begin{frame}{Character Tables}
    \large
    The \textbf{character table} of a group lists the characters of all its \textbf{irreducible} representations
    
    \vspace{1em}
    \begin{itemize}
        \visible<2->{
            \item One row per representation
        }
        \visible<3->{
            \item One column per conjugacy class
        }
    \end{itemize}

    \normalsize
    \visible<3->{
        {\hspace*{\fill} \cite{fulton2013}}
    }
    
    \large
    \visible<4->{
        \todo{Different example, more detailed}
        Example for $\Z/2\Z$:
    }
    
    \visible<5->{
        \centering
        \begin{tabular}{r | c c}
                    & 1 & 1 \\
            $\Z/2\Z$& 0 & 1 \\ \hline
            triv    & 1 & 1 \\
            sgn     & 1 &-1
        \end{tabular}
    }

\end{frame}
